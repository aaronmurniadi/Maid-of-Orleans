\chapter{The Conspiracy}

\drop Four leagues distant from Cambray\footnote{A town in the
  department of Nord, France, famous for the manufacture of cambrics,
  which take their name from it.} the towers of Beaurevoir Castle rise
from forest-crowned heights. In selecting this spot the builders
combined the useful and the beautiful, for the castle was famous both
for its strength and for its attractive situation. The view from the
upper windows and from the towers repaid the appreciative observer at
any season of the year, but he would have lingered longest in admiration
when park and gardens, wood and meadow, field and grove were decked in
the beauty of early spring, when the thickly clustered villages, east,
west, and north, smiled amid their luxuriant crops, or when on the
southern heights the Argonne forest was clad in its most gorgeous
greenery. How much more attractive the beauties of this spot must have
been to a child whose greatest delight was to be among the flowers of
the garden and meadow, the birds in the parks, and the varied scenery!
How closely such a child must have been attached to such a spot! How
strong its temptation to pass all its time with nature!

Just such a child as this had been allured to the park and gardens by
the sunshine of an early April day in the year last named,---a girl
blooming with color, vigorous with health. At a distance she appeared to
be about eighteen years of age, but closer observation showed she could
not have been much over fifteen. Of all the beautiful things in this
beautiful scene she was the most attractive, as she frolicked and
skipped about like a fawn, bounding over the flowery meadows for the
first time. As she ran about in the sunshine she gave expression to her
childish joy at each fresh manifestation of the marvellous work of
spring, and broke out in most exultant exclamations when she discovered
the first violets in the grass.

Two ladies slowly following her, and engaged in earnest conversation,
were attracted by her outcries. ``There now,'' said one of them, a
somewhat slender person with angular features and sharp eyes, ``you see
what an undisciplined creature she is. Is it proper for her to behave in
such a manner? This comes of letting her have her own way. How often
have I protested! But of what use is it? When you see that your talking
is of no avail it is best to hold your tongue. If you do not, then they
say, `Oh, yes, that's the way envious old spinsters always talk.'\,''

The other lady, whose handsome face, beaming with good nature, was in
striking contrast with that of her companion, cast an appealing glance
at her. ``Oh, dear Rosette, are you not mistaken? Who would dare to
insult my husband's sister by making such a remark?''

``Oh, well, you know people often think many things they do not say.''

``That is true. But even if they do, why should you conclude they are
thinking things about you they do not venture to say?''

``I cannot give you any precise reason.''

``Then I must tell you it is not kind to think evil of others,
especially of your own friends, unless you have sufficient cause to do
so. But never mind. You were speaking of Marie. You are offended with
the behavior of the poor child.''

``Child! A fine child she is,---ha! ha! You ought to have known some
time ago that she is no longer a child. She is a grown-up girl.''

``Let us hope she may not discover it for a long time yet. How happy she
would be if she could always preserve her childlike nature! Look at her,
dear Rosette! Is it not a beautiful sight---such an innocent child,
sporting in pure delight?''

The sister-in-law turned up her nose.

``But why is not her behavior proper?'' continued the other. ``Proper!
What is proper? Are not many things proper which are called highly
improper? Marie is in her own world here. She has grown up in it, is
attached to it, and enjoys herself in it. You cannot imagine how
delighted I am to see her thus. Poor little one! Orphaned at an early
age, she has never known the comfort of a father's or mother's embraces,
and shall I begrudge her her harmless pleasures?''

``It would be much better if she were to begin leading a more quiet and
serious life right away, in preparation for her future.''

``What has the future in store for her?''

``Is she not intended for the convent?''

``Who says so? She is sole heir of Louis of Chafleur, who has left her a
rich property. Why should she take the veil?''

``She will not take it voluntarily. I think it is the wish of your
husband.''

``I think you are mistaken. At least, I do not know of any such plan.
John simply said that a convent would be the safest retreat for Marie in
case the tumult of war should invade the Argonne forest. To seek the
shelter of a convent and to take the veil are two different things.''

Rosette's eyes glistened with malicious triumph as she looked at Marie,
who at that instant came bounding forward with a bunch of violets and
put an end to the conversation; her look seemed to say, ``I know some
things better than you.''

While this was going on in the park, two men were standing at an upper
window of the castle. They were considerably beyond middle age, and
resembled one another in a certain cold, crafty, calculating expression
of countenance. One of them wore the usual costume of a knight, the
other the conventional dress of a high church dignitary. One was John of
Luxemburg, lord of the castle; the other, Pierre Cauchon, Bishop of
Beauvais.

``The girl is really a handsome child,'' said the Bishop, as he looked
at Marie.

``Oh, yes,'' slowly assented the lord of the castle. ``But,'' he added
with a peculiar twinkle of the eye, ``I know something that is more
beautiful.''

The prelate understood. ``Hm! I won't dispute that. These are fine
possessions. It would be a pity to have them pass into the hands of
strangers.''

``You have echoed my very thought, your reverence. So I think we are
agreed on the general point.''

``You mean that in these times of disturbance there is no place where
Marie will be so secure as in the cell of a convent.''

``Exactly, and unless I am mistaken that is also what you mean.''

``In a general sense, yes; but we have not yet considered the most
important point.''

``Let us come to it.''

``The question arises, How is the girl to be secured for the convent?
and next, How is she to be taken there?''

``I will see that she is taken there. As to the rest of the business, I
appeal to the experience of your reverence.''

``Hm! a difficult task when, as in this case, the novice has the utmost
aversion to a convent.''

``It is not so difficult as appears at first sight. I know of similar
cases where the task has been successfully accomplished.''

``Yes, but under peculiar circumstances.''

``The circumstances in our case are similar.''

The Bishop's face wore a crafty expression. ``That is truly quite
another thing. Let us hear about it.''

``Of course Marie's property remains in possession of her guardian until
she reaches legal age, when it is at her disposal.''

``That is clear. But what will the Church get?''

``Patience, your reverence. If she should not reach that age---and that
is not impossible---''

``Well?''

``I understand that in such a case the property is legally mine.''

``That is also clear. But what will the Church get?''

``In that case we can make an agreement as to how much the Church shall
have.''

``We understand each other, noble knight. But supposing she reaches
legal age?''

``Then the Church must see to it that the legal requirements are not
binding. I say `legal requirements.' You understand me, holy father?''

``Perfectly, my noble friend. Sometimes we have had to grant exemptions
from requirements which afterwards were shown to have been void because
of irregularities.''

``I am glad we understand each other so well.''

``Yes, but what will the Church get?''

``The same as in the other case, namely, a share of the property, only
the Church will not come into actual possession until after the death of
the testatrix.''

``Hm! It seems to me, my noble friend, that you not only propose to take
the lion's share, but the entire prize. The Church would have the first
claim in case of death.''

``You haven't let me finish, your reverence. Until the death of the heir
I will secure you, as the representative of the Church, a yearly income
of three hundred pounds.''
