\chapter{In Camp and Court}

\drop News of the siege of the City of Orleans by the English at last
reached the village of Domremy. No one was more deeply affected by it
than Joan, for she believed from what her confessor had told the
villagers that with the fall of Orleans the King's cause would be lost,
that there was no hope for the raising of the siege, and that the
wretchedness of the fatherland would then be complete.

Scarcely had Joan heard the news before she left the village to meditate
upon this new situation in some one of her favorite solitudes. She was
at this time about seventeen years of age, blooming and beautiful in
person, but unchanged in nature and habits. She longed to abandon
herself to her thoughts and impressions in solitude as she used to do
when tending her father's flocks. Deep down in her heart she felt the
sorrows of others now as she did then, and was moved by the same
irresistible desire to help them. She longed to prostrate herself before
her saints, to look into the clouds with supernatural vision and see
their figures and hear their voices as she used to do. Her communion
with the spiritual world at this time had become so intimate that she
could question her saints and hear their instant replies. The Fairy
Tree, under which she fed the birds, the miraculous spring where the
fawns frisked about her, and the chapel at the cross-road near the oak
forest, in which she had most of her visions, were her favorite resorts.
In this chapel she knelt before the image of Saint Catherine,
unconscious of the outside world. The burden of her fervent prayer was
the necessities of the country, the rescue of the City of Orleans, and
the coronation of the King.

``O that I were a man! O that I were a commander!'' she sighed. ``I
would rush to the rescue. Perhaps it is not impossible. Does not the
wolf fly from me when my saints are near? Can I not hide my maiden's
figure in the garb of the soldier? Are not these limbs strong enough to
wear armor? What if the dear saints should commission me to rescue the
fatherland!''

Absorbed in such thoughts and longings, she lost herself in communion
with the celestial world, and in a vision she saw her favorite saints in
the glowing clouds.

``Why do you tarry, Joan?'' said the voices. ``Cities and villages are
being destroyed every day. Daily the blood of the people is being shed.
Arise! Execute the decree of Heaven.''

``But,'' said Joan, ``how may I know it is Heaven which sends me?''

``The signs of your mission will not fail.''

``And what is my mission?''

``To raise the siege of the city of Orleans, and conduct the King to his
coronation at Rheims.''

``How shall I begin?''

``Go to the King and offer yourself to him as commander of the army.''

``To whom shall I apply so that I may reach the King?''

``Go to the knight, Robert of Baudricourt.\footnote{Robert of
  Baudricourt was the governor of Vaucouleurs.} He will help you.''

Joan returned home, and remained several days deeply absorbed in
contemplating the mission to which she had been assigned. She would
often steal away to her little chamber and weep bitterly; for although
she felt exalted by the heavenly decree, still, it seemed impossible for
her secretly to leave all the dear ones at home,---father, mother,
brothers, and sister. And yet she must go secretly, for her father never
would approve of her purpose or consent to her going, and no other way
suggested itself. They had grown so accustomed to seeing her absorbed in
silent and solitary meditations that they kept aloof from her at such
times. It had been village gossip for years that she communicated with
spirits and practised magic. In what other way indeed could her mastery
of the wild beasts be explained? Her brother Pierre, however, who was
devotedly attached to her, was an exception. He never pained her by
suspicions. She had no secrets from him, and she came to him now in
perfect confidence and wept upon his breast.

``It is not true, Pierre,'' she said, looking up at him with her
beautiful tearful eyes, ``that you mock at me as the others do?''

``How can you think such a thing of me, little sister?''

``Oh, I do not think it, my brother.''

``And yet your question seems to imply that you do.''

``Not at all, Pierre. I know very well that you love me, but you must
tell me so over and over again. I know very well you do not mock me, but
even that does not satisfy me. I must have the assurance from your own
lips.''

``I know very well, Joan, that you are a favorite with your saints, that
they manifest themselves to you in the clouds, and that you talk with
them as you talk with us.''

``Yes; you believe me when I tell you these things. But when I tell the
others---''

``Oh, my sister, they do not know you as I do. I know that you never
speak an untruth.''

``And yet my actions now must be deceitful. Alas! Pierre, that is what
distresses me.''

``But remember, little sister, that you are obeying the celestial ones,
that it is the fatherland which calls you.''

``And still it grieves me, my brother. I go about here just as usual.
Father, mother, and all the others think that I shall always go on this
way, and I let them think so, and purposely strengthen this belief while
I am preparing to leave them secretly. Oh, Pierre, they will never
forgive me.''

``Why should you distress yourself with such thoughts, my sister? You
know that you must undertake this mission. And it is right you should,
for the will of Heaven is superior to the human will. When father and
mother and the others hear what Heaven has accomplished through you, do
you not think they will forgive you?''

``Your words have done me good, my brother,'' cried Joan, her clear,
brilliant eyes shining with happiness. ``Would that I could always have
you by my side and hear your voice! If you were near I would fear no one
whom I may encounter.''

``I will go with you, my sister.''

``No, Pierre, you cannot.''

``And why not?''

``Is it not enough for me to bring sorrow to our parents? Would you add
to that sorrow by secretly going away also?''

``You are right. I ought not to go. You are obeying the decree of
Heaven, but I cannot offer that plea. But I know of some one who might
go with you.''

``Who?''

``Uncle Laxart. He also loves you, and he will not have to ask
permission of any one.''

``But will he go?''

``I will speak to him about it.''

The next day (in the year 1429)---it was the day of the Three Holy
Kings---Joan crossed the snow-covered valley to the Fairy Tree,
sprinkled crumbs for the birds as usual, and listened to their grateful
songs. Soon afterwards she was lost in deep reverie in the chapel at the
cross-roads, and while in this state her enraptured eyes beheld her
saints, Catherine and Margaret, in the clouds.

``The hour has come, Joan,'' she heard them say. ``Arise! the Queen of
Heaven will be with you.''

``But I must go all alone,'' she replied. ``They will call me an
adventuress.''

``Not so! Your protector is already at the door.''

As Joan arose she saw a man approaching the chapel. With joyous surprise
she recognized her uncle, Duram Laxart.

``I know all, Joan,'' he exclaimed. ``I am ready to escort you as soon
as you need my protection. I have already been to Vaucouleurs and have
seen the knight Baudricourt. Start as soon as you can get ready. We will
lodge with Wagner, whom you know.''

Before the astonished maiden could reply her uncle was off in the
direction of Vaucouleurs.

The time for departure had come at last. Deeply agitated, she stood at
the door of the chapel, and looked once more with tearful eyes out over
the valley. Once more her gaze lingered upon the miraculous spring, the
Fairy Tree, and her home at Domremy, and her soul was filled with tender
and sacred associations.

``Farewell, O Wonder Tree, where I have spent so many happy hours,'' she
said between her sobs. ``And you, little birds, farewell! Alas! Joan can
never feed you again. In vain will you wait for her. Farewell, dear
spring, whose music I have heard so often in my happy dreams. Tell the
deer I cannot play with them again. Farewell, loved valleys and fields!
How happy I was when I played here with the companions of my childhood!
Alas! I shall never see you again! Farewell, my father! My beloved
mother, farewell! And you, my Pierre, my good, dear brother. Oh, how
hard it is to leave you! Alas! never again shall I look into your true
eyes, never again hear words of love and sympathy from your lips.
Farewell all, all farewell! Grieve not that I leave you. Be not angry.
It cannot be otherwise. No! it must be so, for Heaven has decreed it,
and the fatherland has called me. Away, Joan, away! The struggle is at
hand.''

No one could have seen the simple peasant maiden at that moment, her
eyes shining as the tears glistened on their lashes, no one could have
realized her strength of will in giving up all that had filled her soul
with sorrow as she thought of leaving it, no one could have watched her
passing down the valley like a soldier defiant of danger, without the
conviction that it was an event fraught with the highest significance
for France.

Joan found her uncle at Wagner's house in Vaucouleurs. He had already
called upon Baudricourt, but was sent away with instructions to reprove
his silly niece and take her back to her parents. Though not in the
least discouraged, Joan spent the night in prayer, and in the morning
went to see Baudricourt. She found him in the company of Jean de
Nouillemport de Metz. Both laughed when they learned the nature of her
errand, but she spoke with such sincere conviction of her celestial
visions that Baudricourt at last dismissed her with a promise to give
the matter serious consideration. Subsequently, when Joan prayed in the
church, and the people came in crowds to see ``the saint,'' a priest
approached her with a crucifix to see if she was possessed of the devil.
Joan fell upon her knees and kissed the holy symbol, and the priest
declared, ``She may be mad but she is not possessed.'' On her way out of
the church she met the knight Nouillemport de Metz, to whom she thus
appealed: ``Alas! No one will believe me, and yet France can be saved
only by me.'' The words reminded him of the prophecy of Merlin. After
observing her more closely, and recognizing her spiritual purity and her
resolute determination of purpose, he expressed his willingness to take
her to the Dauphin, and he had little difficulty in persuading
Baudricourt to join him. A few days afterwards Joan was delighted to
find herself on the way to Chinon with the knights and their men at
arms. In her costume she looked like a slim, handsome page rather than a
trooper. Chinon was more than one hundred and fifty leagues away, and
for half that distance the country was occupied by the English. Hence
they were obliged to make wide circuits, and frequently halt in the
forests and ford rivers. After a fourteen days' march they reached the
city of Gien\footnote{Gien is in the department of Loiret, and
  thirty-eight miles in a direct line from Orleans. Its principal
  industry is the manufacture of faience.} on the Loire. The news spread
like wildfire that the Maiden who, according to Merlin's prophecy, was
to rescue France, had come, and all hastened to extend her an
enthusiastic welcome.

After leaving Gien there was little danger, and at last they safely
reached Chinon and put up at an inn. Here, as at Gien, the news of
Joan's arrival spread rapidly, and attracted a great crowd. To satisfy
the universal curiosity, she appeared on the balcony and was welcomed
with enthusiastic shouts. Her knightly companions promptly waited upon
the Dauphin; but they found him greatly discouraged and in a despondent
mood because of the news that the Englishman, John Falstaff, had
repulsed the French, who tried to prevent him from taking supplies of
herring to his countrymen before Orleans. The Dauphin's disappointment
over the ``herrings day'' defeat, however, would have been short-lived
had he not at the same time been overtaken by a calamity which seemed to
him even worse, namely, his utter lack of money and the consequent
emptiness of his kitchen and cellar. In such a mood Joan's companions
found him. At first he listened to them with indifference and a
contemptuous smile, but when they told him the people had recognized the
Maiden as a saint, and welcomed her as the rescuer of France, it
occurred to him she might be instrumental in relieving his necessitous
condition. At last he ordered that she should be admitted. To test the
prophetic gift ascribed to her, he received her standing among the
nobles of his court, while another person sat on the throne.

Joan recognized him at once, however, and advancing to him, knelt, and
greeted him with these words: ``God grant you a long and happy life,
Dauphin.''\footnote{Joan called him ``Dauphin'' because she did not
  consider him a king until he was crowned.}

``You are mistaken,'' he replied. ``Yonder is the King,'' pointing to
the person on the throne.

``Noble prince,'' she answered, ``you cannot deceive me. You are the
Dauphin.'' A murmur of astonishment ran through the hall.

``Sire,'' she continued, ``if we can be alone I will tell you something
that will remove all doubt as to my mission.''\footnote{The doubt which
  was thrown upon the King's legitimacy greatly weighed upon his
  spirits. This doubt Joan removed. Her words to him are thus reported:
  ``On the part of my Lord, I tell thee thou art true heir of France and
  son of the King, and he sends me to lead thee to Rheims to the end
  thou may'st receive thy crown and thy coronation if thou wilt.''}

The Dauphin conducted her to the adjacent oratory, and there, according
to the tradition, she revealed things to him which he was certain none
could know but God and himself. He was so sure of this that at the close
of the interview he exclaimed: ``I am convinced of your divine
commission, but my councillors must also be convinced.''

``Very well, sire,'' she replied. ``Summon the three most learned and
experienced to meet me in the morning, and I will give them a sign.''
Her wish was gratified. The three selected were the Archbishop of
Rheims, Charles of Bourbon, and De la Tremouille, the King's minister.
They first required her to give her history, and then they asked for the
sign. Joan went back to the oratory. Then, according to tradition, the
heavenly ones appeared, and with them an angel in long white raiment.
The latter carried a brilliant crown and slowly advanced into the
audience-room.

``Sire,'' said the angel, ``trust this maiden whom Heaven sends to you.
Give her at once as many soldiers as you can raise. As a sign that you
shall be crowned at Rheims, Heaven sends you this token.'' Thereupon the
angel handed the crown to the Archbishop, went out as he had entered,
and disappeared through the ceiling of the oratory. So says the
tradition.

The three councillors were not yet fully satisfied, however. They
suggested that Joan should be examined by the learned theologians of the
University of Poitiers.\footnote{Poitiers is the capital of the
  department of Vienne, and is famous not alone for its university, but
  for its cathedral and the Temple de St.~Jean, the oldest Christian
  structure in France.} When they also asked her for a sign, she
replied: ``Give me soldiers and you shall have signs enough.'' They
finally reported that she was trustworthy, and that the King ought to
accept her service. The Dauphin's council promptly decided to raise as
many troops as possible, place the Maiden in command of them, and send
her with a convoy of supplies to Orleans. In these few days popular
sentiment had changed rapidly, cheerful self-sacrifice and enthusiastic
eagerness for action took the place of discouragement and dissension.
Knights and their men at arms offered their services, and wealthy
burghers sacrificed their treasures for the cause of the country. The
Dauphin at last was also in a cheerful frame of mind, for his treasury
was filling up and he could once more take some pleasure in living. He
was also in a position now to be of service to the Maiden. He presented
her with a general's outfit,---a master of horse, two pages, two
heralds, and a chaplain.

About this time the Duc d'Alençon\footnote{The Duke d'Alençon was a
  relative of the King, and had been held prisoner by the English for
  three years. He was released upon the promise of a heavy ransom.}
returned from English captivity. He noticed with great delight that
every one was eager to follow the Maiden into battle. He immediately
mortgaged his property, purchased war equipment, and accepted the duty
of preparing the convoy of supplies. Joan met with an affectionate
welcome from his wife, who had come to Blois, where the preparations
were going on.

The twenty-sixth of April, 1429, was fixed as the day of departure. Joan
had previously sent her herald Guienne with a letter to the Duke of
Bedford\footnote{The Duke of Bedford, an English general and statesman,
  was John Plantagenet, third son of Henry IV, and at this time regent
  of France. He was conspicuous in the prosecution of Joan of Arc.},
which she had dictated to her chaplain. It ran thus:

\begin{quote}
\begin{center}
\textsc{\normalsize Jesus, Maria}
\end{center}

“King of England, account to the Queen of Heaven for the blood you
have shed. Surrender to the Maiden the keys of all the good towns you
have captured. She offers you peace in the name of God if you make
reparation and honestly return what you have taken. If you fail to do
this she will everywhere attack your troops and drive them out of the
country. And you, archers and soldiers before Orleans, go quietly back
to your own country, or protect yourselves against the Maiden. France
has not been given by Holy Mary’s Son to you, but to the true heir,
King Charles, who will enter Paris in good company. You shall see who
has the better right, God or you, De la Pole, Count of Suffolk, Talbot
and Thomas, who have taken the field for the Duke of Bedford, the
so-called regent of the Kingdom of France for the King of England. If
you do not leave the city of Orleans peacefully, Duke of Bedford, you
will force the French to achieve the most glorious exploit ever known
in Christendom.”

\begin{center}
“Written on Tuesday in Passion Week.”
\end{center}
\end{quote}

This letter, however, never was answered. The herald did not come back.

On the day appointed the expedition set out from Blois. At its head was
a procession of priests singing hymns, Joan's chaplain leading them with
his banner. Next followed the leaders, Duc d'Alençon, Marshal de Retz,
Admiral de Coulent, De la Maison, Laval, Potou de Saintrailles, Count
Dunois and La Hire, in whose retinue was Jean Renault. Then came two
hundred horsemen, and a long train of wagons loaded with supplies
brought up the rear. Joan in full armor, wearing a shining helmet which
covered her closely cropped locks, and carrying a sword whose hilt and
scabbard were ornamented with lilies, rode among the leaders. Upon one
side of her banner, which was thickly sprinkled with lilies, was a
picture of the Saviour with the orb in His hand and an angel on either
side of Him; on the other, the inscription, ``Jesus, Maria.''\footnote{Joan
  of Arc, testifying at her trial, said: ``I had a banner of which the
  field was sprinkled with lilies; the world was painted there, with an
  angel at each side; it was white, of the white cloth called bocasine;
  there was written above, I believe, `\textsc{Jesus, Maria}'; it was
  fringed with silk. Because the Voices had said to me, `Take the
  standard in the name of the King of Heaven,' I had this figure of God
  and of two angels done. I did all by their command.''} Her demeanor
was serious and dignified, serene confidence shone in her beaming eyes.
Her only regrets were the profanity of the soldiers and La Hire's loud
prayer every morning and evening: ``Dear God! do for La Hire as he would
do for Thee if he were the dear God and Thou wert La Hire.''

On the third day they were before Orleans, but the city was on the other
side of the Loire, and there was no bridge. They occupied a redoubt on
their side of the river, which the English had abandoned because it was
of no use to them. At this juncture the Bastard of Orleans,\footnote{Count
  Jean Dunois, called the ``Bastard of Orleans,'' was born in 1402, and
  died in 1468. He was the natural son of Louis, Duke of Orleans, and
  Mariette d'Enghien, and at this time was in command at Orleans.}
commander of the city, came in a barge to meet them. By his advice they
went two leagues farther up the river and made a halt near Castle Chécy,
where they found a French garrison. Count Dunois agreed to send a fleet
for the transportation of the supplies, but at three in the afternoon it
had not come. The sky was overcast, thunder growled in the distance, and
the waves of the Loire were lashed by fierce winds. The courage of the
soldiers began to waver.

``When this storm subsides,'' said the Duc d'Alençon, ``the English
vessels will be here instead of ours, and then all will be lost.''

``Ah, you forget,'' said the Maiden, ``that I promised you in the name
of God we should enter Orleans successfully.''

``H'm! it does not look as if you could keep your promise,'' replied the
Duke.

``Have a little patience,'' said Joan, as she closely scanned the sky.
``Before a quarter of an hour passes the wind will change.'' She retired
a little distance to pray, but hardly had she knelt before a favoring
wind sprung up and the vessels which had been detained by the storm
arrived.

``Now what do you think of that, Jean?'' said La Hire, as they began
loading the supplies.

``I think, noble sir,'' replied the youth, ``that the Maiden in her
pastoral life has had ample opportunity to observe the wind and weather,
and is therefore able to predict changes like these.''

``Oho! Then she is an impostor!''

``Why so, noble sir?''

``Do you not understand? Does she not make people believe that the winds
change in answer to her prayers?''

``Oh no, certainly not. She does not pray on account of the wind. She
prays because prayer is a necessity to her, because of the impelling
forces of her nature, and because she feels happy in communing with
Heaven. Her special prayer is for strength and help from on high for her
great work, which is beginning this very hour.''

``H'm! But she deceives the multitude by it, just the same.''

``She does only what she must do. Does the sun lave itself every evening
in the sea just because the people believe it does?''

``I am not criticising you, my young friend, but one minute you deny the
supernatural in the manifestations of the Maiden, and in the next you
extol her to the very skies.''

``What are wonders anyway, noble sir? What the blind multitude regards
as a wonder easily resolves itself into harmony with nature to the
reflective person, and what the multitude passes by without observing at
all is a wonder to the intelligent thinker.''

``Explain yourself more clearly.''

``As to the first point, the Maiden herself is a sufficient
illustration. Do not these people recognize a wonder in this change of
wind, while you see nothing at all extraordinary in it? As to the other
point there are a thousand illustrations. The sky with its stars, the
flowers of the field, the worm in the dust,---all these are wonders of
creation which the multitude scarcely notices, but which are marvellous
to the observant thinker.''

``And this Maiden?''

``She is a wonder in both ways, and therein lies her extraordinary
power. She is believed to be a prophetess who has direct communication
with Heaven. The people regard her as an actually divine wonder, because
of her purity of heart, her celestial confidence, her unsullied
patriotism, and her spiritual illumination. Indeed, noble sir, the
Maiden is a wonderful gift of Heaven to stricken France.''

``Then you also believe in her success apart from her divine
commission?''

``I do not dispute her divine commission. She is executing it because
the divine voice in her own heart has charged her with that duty. Do I
believe in her success? Look at these people! How their eyes are fixed
upon this Maiden! At her command of `forward' they would plunge into the
Loire and follow her, believing that its waters would bear them up. Will
you not yourself, noble sir, although you do not believe in her divine
commission, gladly draw your sword when the lily banner waves before
you? If the spirit which Joan has roused in our little band is animating
all France, how can we do otherwise than expect success?''

``You are right, my young friend,'' said La Hire, extending his hand.
``I thank your gallant father in his grave for the training he gave you.
Yes, yes, it must be so,---when religious or political enthusiasms fire
a people, great results always follow. In this case it is a joint
enthusiasm. The victory will be ours, and we shall thank the Maiden for
it. I will not again grieve her with my prayers. When it is prayer time
I will go so far off that she cannot hear me. But one thing more, Jean.
Have you heard anything about Marie?''

``Alas! noble sir, I have not. As you well know, I have not ceased
making inquiries, but as in your case, the turmoil of war has prevented
me from obtaining personal information.''

``Yes, yes, I know. I cannot tell you how that poor child's situation
troubles me. I have waited from week to week for an opportunity to speak
a few words to this lord of Luxemburg and his bishop''---and his grip of
his sword indicated what kind of words he had in mind. ``But let us
hope,'' he resumed, ``that the Maiden whom we serve may open the way to
Marie's release. First, we must send the English to the devil. After
that nothing shall prevent me from finding Marie, and''---casting a
significant glance at Jean---``I know who will stand by me.''

``To the end of the world, noble sir,'' cried the youth, his flashing
eyes showing that the words came from his heart.

``Good, good, I am sure of it; but it is time that we were off.'' He
pointed to the last of the vessels, which was held for the troops. Soon
they came to the line of the English intrenchments, which stretched
around the city.

``Well,'' growled La Hire, ``these English gentlemen do not like to show
themselves, and yet it would be an easy matter to break these nutshells.
I wonder if they have run away.''

``They are there, and can see us,'' said Jean. ``They are lying behind
the walls, but they have no ordnance to use against us.''

It turned out as Jean said. The English made no assault, and the little
flotilla reached the city unharmed. There was unbounded enthusiasm when
the Maiden appeared with her banner at the gates. The people would have
carried her in their arms had not the commander of the city forestalled
them by having a horse in readiness for her. She mounted and rode in
triumph to the cathedral, where a \emph{Te Deum} was sung, the first
which had been heard for a long time within its walls. Then she was
escorted to the house of Jacques Boucher, treasurer of the Duke of
Orleans, where she was to lodge. Now for the first time she put off her
armor, drank a cup of wine diluted with water, and then withdrew with
the wife and daughters of her host to her chamber. There was lively
commotion in the streets until far into the night. All anxiety
disappeared on that 29th of April, 1429. An old chronicle relates that
the people and the soldiers believed an angel had come down from heaven
to them. To the same extent that their despair had vanished and given
place to joyous enthusiasm, courage waned in the English camp. Most of
those brave soldiers, particularly the spearsmen and archers, believed
the Maiden was either a messenger from heaven or from hell, either a
saint or a mighty magician. Their leaders inveighed bitterly against the
Dauphin because he employed unknightly weapons, weapons of hell.

On the next day Joan urged an immediate attack, but at a council of the
most experienced leaders it was decided to wait at least for the arrival
of the next contingent of troops from Blois. They did not altogether
believe in Joan's divine commission, but they thought it best to take
advantage of the popular enthusiasm which she had aroused. Count Dunois
returned to Blois to hasten reinforcements forward, and on the fourth
day his banner was seen on the left bank of the Loire. His route led
directly past the English encampment. Joan could no longer remain
inactive. ``We must go out and meet them and fetch them in,'' she cried,
at the same time mounting her steed, seizing her little battle-axe and
banner, and riding to the gate. The knights shook their heads. No one
was eager to rush directly into the jaws of the lion, for it did not
seem possible that any one would come back if the English came out and
attacked them.

``Now,'' shouted La Hire, ``no one shall say that La Hire has been
outdone in courage by a woman. Forward,'' he commanded, galloping after
the Maiden, and followed by his little band. Count Dunois had halted
some distance away, evidently awaiting help from the city. As soon as he
saw the banners of the Maiden and La Hire he moved forward along the
first line of intrenchments. The two forces soon met and advanced
towards the city along the very front of the English camp, but such was
the Englishmen's fear of the Maiden that not one of them ventured out.
No one even hurled a missile. They looked on quietly, as the little band
passed their lines and safely reached the city. As further
reinforcements were on their way, it was decided on the following day to
attack Fort Saint Loup.

Early in the morning, while Joan, wearied by her exertions on the day
before, was still sleeping, some of the captains sallied out with their
troops and made a furious assault upon the fort. The English, seeing
only their customary assailants, fell upon them, and after a hard
struggle beat them back. At that instant the Maiden, bearing her lily
banner, rode to the Burgundian gate. ``Halt!'' she shouted to the
fugitives. ``Look you, the Maiden whom God sent to you is here. Follow
me to victory.'' At once she plunged into the thick of battle. Her
presence acted like magic on both sides. The French impetuously followed
her, Daulon, Master of horse, La Hire, and two other knights, leading
the first charge. The English wavered.

``Why do you hesitate?'' cried Guerard, their leader. ``Shame and
confusion to any one who fears this country girl! Drive her back to her
village and her father's cows.''

His appeal was unheeded. The soldiers stood for a moment staring at the
banner in the hand of ``the witch,'' and then, as if at the word of
command, rushed in a panic for the protecting walls of the fort. ``On,
my brave ones, forward to the battle and victory,'' cried Joan, as she
furiously galloped after the fugitives.

``Now, soldiers of France!'' said La Hire, ``she is doing more than her
share. On, my children! Shall we let this brave one do all the work
alone?'' He spurred his steed, but his heavy battle horse could not
overtake the Maiden's light courser. The next instant a single knight of
La Hire's troop flew after her, and in a few seconds his sword was
waving at her side. It was Jean Renault. Enthusiasm such as he had never
felt before had seized him. He was oblivious to all sense of danger.
Scarcely was the last Englishman through the fortress gate before the
Maiden and Jean rushed through also. The astonished English soldiers saw
the lily banner in their very midst. Before they had recovered from the
deadly fear it inspired, it was flying on the wall. The French poured
through the gate, and victory was soon complete. Those who resisted were
cut down, and the rest were taken prisoners. Some of the fugitives had
fled to the tower of the church within the walls, but these unfortunates
were either killed upon the steps, or hurled themselves from the
windows. A more fortunate remnant came out of the sacristy, where they
had arrayed themselves in the robes of the priests. These were greeted
with jibes and laughter as they begged of the Maiden to be made
prisoners. Amid the peals of bells and the triumphal shouts of the
people, the Maiden entered the city at the head of her soldiers.

Three days after this---a festival day intervening---the leaders decided
to make a feint upon the right bank, covering an attack upon the left.
As Orleans lies upon the right bank of the Loire, the commander of the
city kept a large number of boats for crossing. In the midst of the
stream, but somewhat nearer the left bank, is an island which the
English had not occupied. The French landed upon this island, Joan and
La Hire, with his troop, in the lead. The boats were fastened together
and thus made a bridge to the left bank, over which they advanced for an
attack upon the first fort, Le Blanc. It would have been easy for the
English to stop the passage, but they did not attempt it. After setting
Fort Le Blanc on fire they fell back upon Fort Saint Augustine.

Joan followed them, and planted her banner half an arrowshot's distance
from the wall. Suddenly there was a shout, ``The English are coming from
Fort St.~Rivi.'' The little band retreated to the Loire, all save
fifteen, La Hire and Jean among the latter. These fell back a little
distance, so as not to expose themselves needlessly to the enemy's
assault, seeing which the English plucked up courage and attacked them,
shouting loudly.

``Follow me,'' cried Joan, waving her banner and advancing upon the
English. The fifteen did not hesitate, rash as the undertaking seemed.
They pressed forward, cutting their way through. When those who had
retreated to the river saw this they came to their assistance, and in a
few minutes the English were driven back into the fort. Joan rushed on
until she reached the palisades, dashed through a breach which Daulon
had made, and planted her banner on the wall. The French rapidly came
up, captured the fort, and burned it. It may well be imagined this fresh
victory was hailed with delight in the city. The bells again pealed as
the soldiers entered, but their reception was a quiet one as compared
with the enthusiastic homage which the Maiden received on her way to her
lodgings.

Though Joan was wounded in the foot during the battle and passed a
restless night, she was again on horseback early in the morning. She
rode to the Burgundian Gate with a little band, and ordered it to be
opened. The keeper would not obey, saying that the leaders had decided
not to give battle that day, and had ordered the gate to be kept closed.
When Joan insisted a tumult arose. The people demanded it should be
opened, and at last opened it by force. With joyful acclamations the
crowd followed their inspired leader to the river. The boats which had
been used the day before were lying there, and served this time to carry
her across. Joan held her horse by the bridle and let it swim after her,
and thus the left bank was reached. A shout of joy from the French who
had garrisoned the captured fort welcomed the lily banner. They came out
to meet her, and Joan placed herself at their head. ``Forward, my brave
ones,'' she cried. ``The victory to-day also will be ours.'' An
enthusiastic shout was the reply as they impetuously rushed on to
assault Fort Tournelles.

This fort, the strongest bulwark of the English, was close to the river,
a drawbridge furnishing the only approach to it. On the land side it was
surrounded by a high wall, which had to be passed before reaching the
fort itself. Its garrison was the very flower of the English warriors,
led by the experienced Glasdale. An assault by a mere handful of troops
without ordnance or storming appliances seemed to the English the height
of madness.

In the meantime the number of the assailants continually increased, for
when the leaders in Orleans witnessed the courageous dash of the Maiden
they realized that they must support her. One after another La Hire,
Dunois the Bastard of Orleans, De Retz, Gaucourt, Gamache, Graville,
Tintey, Villars, Chailly, Couraze, D'Illiers, Thermes, Gontaut, Eulant,
Saintrailles, and others appeared upon the scene. By ten o'clock the
assault was general. The French hurled long spears. The English
brandished leaden maces and iron battle-axes and hurled beams, stones,
boiling oil, and molten lead upon the heads of the assailants. After
three hours of furious fighting the French fell back.

``Courage,'' cried Joan, whose banner was always in the front. ``Courage
in God's name. The victory is ours.'' She rushed to a ladder and
ascended. ``Surrender!'' she shouted to the English, ``or you will be
massacred.'' The reply was an arrow, which pierced her shoulder so that
it protruded five inches out of her back. She gave a cry of pain and
came down to the trenches. The English rushed upon her furiously, but a
hand was stretched out to her at once. A heavy battle-axe struck her
protector down. It was the brave Gamache who had come to her rescue. In
a trice other heroes were on the spot, and the English fell back. They
bore the maiden tenderly away and took off her armor. She looked up with
tearful eyes, but they were fixed upon heaven, as was her wont.

``How is it going, Count Dunois?'' she asked.

``We have ordered a retreat,'' he replied, whereupon she partly sprang
up, seized the arrow with both hands, and pulled it out. ``Let there be
no retreating,'' she urged. ``Quick, my armor.'' In a few minutes she
mounted her steed and galloped through the flying ranks. ``Halt!'' she
pleaded. ``Have courage, in God's name. In half an hour the English will
be in our hands.''

The effect of her heroic resolution was wonderful. The soldiers turned
back with cheers. Daulon grasped the lily banner and carried it to the
wall. Joan hastened forward and again led the assault. The terror of the
English at the reappearance of the Maiden cannot be described. They had
believed her dead. They were certain now that she was in league with
Satan. They dropped their weapons and fled, and fear lent wings to their
flight. Loud cries of horror from the water side completed the disasters
of the day. An attack had been made upon the drawbridge. Glasdale had
hastened there to protect the weak point. A shot fired by Daulon
shattered the pier, and the bridge with all its defenders fell with a
crash into the Loire. Glasdale, weighed down by his heavy armor, was
drowned. It was this disaster which had caused the outcries. The day
ended in a tragedy for the English. ``Save yourselves as you can,'' was
the signal for flight. The fort was taken.

In Orleans the bells rang welcome to the troops. They rang the whole
night long in celebration of the victory. The churches were thronged,
and from thousands of grateful hearts rose the \emph{Te Deum laudamus}
to heaven. The next morning dense smoke ascended from the English camp.
Suffolk and Talbot had abandoned the siege, set fire to their camp, and
retreated with the remnant of their army.

Thus in nine days Joan accomplished the first part of her mission.

\threeast
