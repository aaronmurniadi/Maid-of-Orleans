\chapter{The Rescue}

\drop Marie of Chafleur had borne her imprisonment with unshaken
courage. She was resolved that she would not be forced to take the vow,
and though she suffered greatly in her damp, gloomy prison,---she who
could still take childish delight in every little flower,---she remained
true to her resolution.

The Abbess, who had been so favorably impressed by her when they first
met, was still more impressed by her firmness, and gave her permission
to visit her. Upon one such occasion the abbess kindly said: ``You
grieve me, my daughter. Your obstinacy may compel me to adopt severe
measures.''

Marie made no reply. She was looking out of the open window at the
garden, which was now in full bloom, and was so absorbed with the view
that she did not hear the Abbess. Her face was all aglow with
excitement, her eyes sparkled, and she gleefully clapped her hands.
``Oh, how beautiful, how beautiful!'' she cried, approaching nearer to
the window. ``Oh, if I could but be among those flowers!''

``You are childish,'' said the Abbess, without manifesting displeasure,
however. ``Listen, and pay attention to what I say.''

Marie wiped away her rising tears and looked into the Abbess's face.
``It is not very long ago that you were as young as I,'' she said,
``and, oh, how beautiful you must have been without that veil! Tell me,
have you never enjoyed yourself in the flowery meadows? Have you never
chased the pretty butterflies, never listened to the songs of the birds,
never breathed the fragrance of the flowers? Oh, tell me.''

``Why do you call up such recollections, child?''

``Oh, yes, I know you have, and so you can understand me when I tell you
it is impossible for me to stay within these walls. I must go. Surely,
noble lady, you will not keep me here any longer. Oh, open the doors and
let me out. I will go on foot and travel through the country all alone
until I find my uncle. And even should I not find him, and have to
suffer hunger, thirst, cold, and heat, still I should be happy. So once
more, noble lady, I implore you to let me go.''

``Child, child, you are asking impossibilities of me.''

``Why impossible?''

``You have no idea of the implicit obedience required of us.''

``But, noble lady, your vows and your discipline only bind you in your
relations to the convent life, not to the outer world.''

``You are mistaken, my daughter. We owe unquestioning obedience in all
things to our superiors. Whatever they demand of us is right. It is not
for us to question or decide.''

``How is that, noble lady? Having dedicated yourself to Heaven, can you
blindly follow human dictation?''

``Child, the will of the Church, to which we bow, is the will of
Heaven.''

``I do not understand that.''

``That is because you are not in the right spirit to understand it.''

``That may be true, but I am sure of one thing.''

``What is that?''

``That you would not poison me even if you were ordered to do so.''

``Child,'' said the astonished Abbess, ``what put such a dreadful
thought as that in your mind?''

``Because, though unconsciously, you have really begun to do it.''

``You shock me! What do you mean? That I would poison---''

``The poison of the prison atmosphere, noble lady, is just as surely
killing me as if it were real poison. So again I implore you to let me
go. Do not degrade yourself by becoming a party to the shameful
conspiracy which has been planned against me.''

The Abbess might have replied to Marie at more length, but she was too
thoroughly convinced of the truth of her words and the injustice which
had been practised toward her to do so; and besides this, Marie's
sweetness of nature and childish ways won more and more not only her
sympathy but her affection.

``I cannot give you your freedom, my daughter,'' she replied, ``but I
will do all I can for you. You may stay in the garden during the day,
but when the Bishop is here you will have to go back to the prison.
Perhaps mildness may accomplish more than severity. It is because of
this hope, bear in mind, that I make this concession. Now go. Here is
the key to the garden.''

Marie fervently kissed her hand and ran off. The Abbess went to the
window and thoughtfully watched her. The joyful expression of her face
showed that her heart approved what her reason and sense of duty half
condemned.

Marie's life now grew more cheerful, for the Abbess kept her word. She
not only allowed her to go daily to the garden, but she admitted her to
her confidence. Of course she had not the slightest idea that this would
induce her to join the order, but she reasoned that if their relations
became intimate she would not suspect any such purpose.

John of Luxemburg all this time was administering affairs as if he were
the lawful owner of Marie's property, and so far ignored all her rights
that after deducting the comparatively small sum due to the Bishop, he
put the rest of the receipts into his own pocket without further
ceremony. It actually seemed as if the two men little by little might
yet accomplish their purpose. Though Marie felt very happy when she
first set foot in the convent garden and the Abbess treated her so
affectionately, yet the roses on her cheeks began to fade, and when she
was alone in her narrow prison during the Bishop's visits her sorrowful
sighs showed she was not in her usual cheerful spirits. Even in the
garden her joyousness would vanish whenever she came near the high wall
which surrounded it. The consciousness that she was a prisoner
embittered every joy, and at last even made the garden unenjoyable. In
this sad frame of mind the scenes of her childhood seemed to her like
bright spots in a lost paradise. As she recalled the happiness of that
paradise, the more keenly she realized the injustice which had driven
her out of it. During that day when all Rouen was witnessing the awful
spectacle in the old market-place, she sat more sorrowful than usual in
her prison. Of course she did not know what was going on, for no news
from the outer world ever found its way within the convent walls.
Whatever the cause may have been, whether her confinement this time had
been longer than usual, or whether she had painted the lost happiness of
her childhood in too lively colors, she was more unhappy than usual.

``My God! My God!'' she moaned, ``hast Thou utterly forsaken me? What
crime have I committed that calls for such a frightful expiation? If I
am guiltless why should godless men triumph? And you, my uncle! Is it
because you are dead that your help is so long delayed? Oh! you brave
one, who all alone confronted those robbers in the forest! Why wait you
so long? Have you been mistaken? Am I not the one for whom you dared so
much? Oh, be quiet, foolish heart, lest I persuade myself I really am
that one.''

She gradually regained her composure, smiled through her tears, and lost
herself in fancies of another kind. At last, scared by her own thoughts,
she resumed: ``O thou Blessed Virgin, protect him! Keep him far away
from here. Those against whom he would contend single-handed are too
strong for him. Protect him.''

As she spoke the last words there was a slight noise at the door. ``They
are coming to let me out,'' she said to herself; ``the Bishop has
gone.'' She wiped away her tears and stepped forward. The door opened,
but it was a man's figure that she saw in the dim light, not the sister
keeper.

``Is it you, Marie of Chafleur?'' the stranger whispered, for he could
see nothing in the prison.

``My God! what is it? Who are you?'' said the terrified girl in a low
voice.

``Be quiet,'' whispered the stranger. ``If you are Marie of Chafleur,
take this bundle. It contains a page's dress. Hasten! I will watch
outside.''

The poor girl trembled like an aspen leaf, but she took the bundle. She
stood for a few seconds as if dazed, but quickly made her decision and
stepped back into the prison. It was some time before she could make the
change of costume, for her trembling hands were not as deft as usual,
but at last she went out into the passage in her disguise.

``Give me all your clothes,'' whispered the stranger, ``for if they are
left here they will betray you.''

Marie fetched them to him, and after making a bundle of them exactly
like the one he had brought, he took the trembling girl by the hand and
led her to the church door. Then he listened. All was still. ``Softly,
softly,'' he murmured as they left the church.

Who can picture Marie's glad surprise as she looked by daylight into the
face of her protector for whose safety she had just before invoked the
Virgin? There was little time for sentiment, however, for scarcely had
Jean closed the door when they heard voices and steps in the street. He
drew Marie down quickly, and they knelt together as if engaged in their
devotions, while he listened intently to every sound near the entrance;
but the steps they had heard were those of passers-by. Jean whispered,
``I believe we have succeeded. Let us thank the Holy Virgin and Saint
Ursula.'' With tremulous voices they murmured their gratitude, and then
Jean said in a low tone: ``Do you feel strong enough, noble lady, to go
on alone?''

``Oh, I will be as strong as a man when away, far away from here,'' she
answered.

``I will take the lead,'' said Jean. ``Follow me at some little
distance, so no one shall suspect we are acquainted with each other. The
whole city is in commotion and crowded with strangers on account of the
execution. They will not pay much attention to us. Do not look around
much, lest some one may recognize you. Keep your eyes downcast, and they
will think you have been overcome by the dreadful spectacle. In this way
we may pass through the gate like the other strangers on their way home,
and after that the Holy Virgin will help us the rest of the way.''

Jean arose and left the church, and Marie followed his instructions.
Everything turned out as Jean had said. The two met many groups standing
on the walks or passing along the streets, and at last safely got
through the gate. Marie could scarcely restrain her exultation, but Jean
went calmly on, hurrying to the forest as fast as she could follow him.
Marie's joy increased as she felt sure that she was rescued, for she
could not believe that a trace had been left which would reveal the
manner of her escape. She looked around, and finding that no one was
following them, she gave expression to her happiness.

``My noble rescuer,'' she said, ``I cannot longer keep silence and
conduct myself like a Capuchin. It is inconsistent with my costume, you
know. I must exult; I must shout, or I shall die right here before
you---''

``Not yet,'' said Jean, without turning round. ``It is not the time for
shouting, still less for dying. We are not safe yet, though the most
difficult part of our undertaking has been accomplished. Your exultation
would be noticed from the city, and then there would be much curiosity
among the pages to find out who it was that was so greatly pleased over
the horrible spectacle. During the next few days they will move heaven
and earth to catch the fugitive. Then some one will be certain to
remember the exulting page of to-day.''

Jean's advice made such an impression upon Marie that she restrained
herself; but when she found herself within the shelter of the forest and
Jean waiting for her, she could no longer keep still. She flew rather
than ran over the green carpet. Her feelings overcame her when for the
first time she found herself in Nature's majestic temple and felt its
subtle and mysterious magic. She fell upon her knees and poured out a
very passion of gratitude to Heaven. She thanked the Virgin for the
happiness of which she had been so long deprived, for her rescue, and
especially for the protection which had been given to her rescuer. Then
she turned to Jean, and her tearful eyes betrayed the emotions of her
heart.

``I have no words with which to thank you, gallant knight,'' said she.

``Oh, my noble lady,'' replied Jean, ``if you only knew how happy it has
made me to have brought you thus far, you would think me recompensed
even too richly. But let us first think of the joy this will bring to
the noble La Hire.''

``What!'' exclaimed Marie, ``are you taking me to La Hire?''

``Yes! But let us hurry on, so that we may get out of the English
district before the news of your flight is spread abroad.''

They went on again, and shortly met the peasants with whom Jean had left
his horse. He bought another for Marie, and they rode off together. Once
more he safely travelled the dangerous road, and on the next day they
had passed the last city occupied by the English. They met with no
difficulties during the rest of the journey, and after changing their
costumes at the lodgings in Chinon, Jean took Marie to La Hire's
apartment.

La Hire was not aware of their arrival at the inn. He was greatly
excited, for he had just heard the news of Joan's death. Aroused to the
highest pitch of fury, he had cursed her enemies, then flung himself
into a chair, and seriously debated whether he should not break his
sword rather than serve such a King longer. He was of too noble a
nature, however, to come to such a decision. There were enemies of the
fatherland yet to fight, and he had some other duties to accomplish. He
had just made his decision, when he heard a well-known voice behind him.

``Here, noble sir, is Marie of Chafleur.''

The knight sprang up. Words cannot describe his joy. He stood like a
statue, with his eyes fixed upon her.

``What!'' he exclaimed at last, ``is this charming girl the little
Marie, my sister's child?'' He opened wide his arms, and she flew to his
embrace. He kissed her hair, and lovingly stroked her cheeks.

``My poor child,'' he gently said, ``how you must have suffered!'' Marie
only answered with a sigh.

``You shall tell me all about it some other time. Be quiet now, my
daughter. From now on no one shall harm a hair of your head. And this
Luxemburg! By Saint George, he shall make reparation to me for every
tear you have shed.''

He resumed his seat, and then turned to Jean.

``Come to my heart, my son. I knew that you were as brave and determined
and valiant as any one, but I did not believe you would bring this child
back. I am anxious to know how you did it, but just now I am too full
under my doublet to listen. I believe there are tears running down my
beard. I don't know when that ever happened before. It must be because
this is a real heart's joy you have given me, my boy. Yes, yes, you and
poor Joan have both shown what resolute purpose can do when it is
persisted in to the end. Children,'' he exclaimed to both of them, ``you
have made me young again. The end will be fine. Just now I determined to
fight the English still longer, and I know,'' with a look at Jean, ``who
will be with me. But that is not the end I mean. That is only the common
duty. I know a finer end than that.'' He looked with joyous eyes from
Jean to Marie, and from Marie to Jean. ``Yes, a finer end than that, and
by Saint George I will accomplish it.''

The valiant knight did accomplish it. Just two years from that day he
stood on the steps of a lordly castle, happier perhaps than he had ever
been before in his life, and watched a carriage which was coming toward
the castle amid the enthusiastic shouts of the peasants. In this
carriage the lawful owner of the castle was making his entrance to take
possession, for the English had been driven out of that whole region.

The master of the castle was Jean Renault, and by his side sat his happy
spouse, Marie of Chafleur.

\threeast
