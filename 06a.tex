\chapter{The Martyrdom}

\drop Jean Renault sat in a tavern at Chinon, abstractedly gazing out
over the flowery fields which were visible from his windows. It was a
day in May, 1431, and the time and the scene painfully reminded him this
was the third spring since the incidents in the forest and the Ursuline
church. He was not a dreamer, however, but a man of quick and resolute
action. It was the thought that he had been prevented from accomplishing
the purpose upon which his heart was set that made him gloomy and
abstracted. A heavy step interrupted his reverie.

``Ha! the villain,'' exclaimed La Hire, as he entered, almost beside
himself with rage. ``The sordid, venal wretch! The dishonorable
scoundrel, who would sell that noble one for contemptible gold! But just
let him wait! I am searching for him and I am on his track!''

``Noble sir,'' interrupted Jean, ``of whom speak you?''

``Of whom am I speaking? Of whom else than Luxemburg? That---''

``Ah! of him! I too was thinking of him.''

``I can well believe it, my boy,'' for although Jean was now a knight,
La Hire continued to call him ``my boy.'' ``I cannot sleep because of
it. Shame and disgrace upon him.''

``I wish we had been at Compiègne. Then we should have had a chance to
meet him.''

``Yes, yes, to meet him---but the poor Maiden!''

``Yes, the poor Maiden! I was also thinking of her.''

``She languishes in a gloomy prison.''

``Yes, in a gloomy prison.''

``Her delicate limbs are loaded with fetters.''

``Yes, loaded with fetters!''

``Condemned to bread and water, like a felon.''

``Condemned to bread and water!''

``She, the rescuer of France!''

``Of whom speak you, noble sir?''

``Of whom do I speak? Saint George, of whom else than Joan?''

``Of Joan? I thought it was of---''

``Ah, you were thinking of Marie! The poor child! May God's vengeance
overtake Luxemburg!''

``And what about Joan?''

``Do you not know? Why, of course you do not, for I have not told you.
He has given her up, sold her to the English, the villain!''

``Who has?'' cried the astonished Jean.

``The Duke of Luxemburg.''

``God help her! And the King?''

``Pah! the King! He doesn't care.''

``Oh, the shame!''

``Oh, the disgrace!''

``But by what legal authority have they put Joan in prison?''

``By what legal authority? Ask the priests who have condemned her.''

``The priests!''

``The vengeful English have given her to the Holy Inquisition. The
Bishop of Beauvais conducted the proceedings, and she has been sentenced
to life imprisonment for heresy.''

``To life imprisonment! But how could they convict her of heresy?''

``They did not convict her. That simple child refuted every charge made
against her by her sensible and devout replies to the questions they
asked her. They condemned her upon the charge of having intercourse with
evil spirits.''

``Shameful, it is shameful!'' cried Jean, springing up in a rage.

``Yes, horrible!''

``Farewell, noble sir.''

``What? Whither go you?''

``To Rouen. You must let me go. I shall not ask the King.''

``But what will you do in Rouen?''

``Summon help if it be possible. Rescue Joan even if it should cost my
life.''

``Would that I could go with you! But I could be of no service. You will
not accomplish your purpose, my boy. They have not only selected a
special tower for her prison, but they have securely bound her with
chains fastened to a post that cannot be reached by you. And there are
two guards constantly on the watch outside and three inside the
prison.''

``But even that, noble sir, does not discourage me. It only makes me the
more eager to be off; and there is something else that urges me on to
Rouen.''

``Well, God go with you, my boy. But I warn you to be careful. I wish I
could go also. I would ask you to wait until I can be there, but it
would not be right. You have waited too long already.''

On the next day Jean rode to Rouen in the disguise of a peasant. While
going through the recovered districts he rode as fast as the strength of
his horse would permit, following the same circuitous route which he
took on his first journey. On the last stretch he made a still wider
detour, which brought him into his own neighborhood, where he met
peasants of his acquaintance, as he had expected. He left his horse with
them and pursued his way on foot to Rouen. That city, as well as its
vicinity, was in the hands of the enemy, and was so strongly garrisoned
that little but English was heard on the streets,---a fact which caused
Jean much misgiving. His appearance, however, did not excite attention,
for intercourse between city and country had gradually been restored,
and the peasants were freely bringing in their products for the market.

Jean's first move was to the church of Saint Ursula. There, at the place
by the wall which was so familiar to him, he fell upon his knees, but he
could not pray. He could hear his heart beating as he listened; but when
he found that he was listening in vain and that there was no sign of
life on the other side of the wall, he became more composed, and prayed
fervently to Heaven for help. Upon his return to his lodgings he passed
himself off for one of the curious crowd which was pouring in from near
and far to see ``the witch.''

``You have come here to little purpose, good friend,'' said his host,
``and yet there are some sights which will repay you. You can see the
cage in which the prisoner was fastened, and the tower in which she is
still confined.''

``Is no one allowed to enter the tower? I would be satisfied if I could
see her even from a distance.''

``Why, what are you thinking of? No one should be allowed to see her,
for she has intercourse with evil spirits! How easy it would be for one
of those spirits to assume the appearance of a peasant and join a crowd
of curious people, just as if it were one of them! Now the prison door
opens! Hush! the spirit gets in there! and ps-t---they are gone. Do you
see? That is the way with witches.''

``Is that so?''

``Oh, yes! My grandmother, blessed---''

At this instant the loquacious host was called out. When he returned he
had forgotten his story in his eagerness to make an announcement to his
guest.

``You are a very lucky man,'' he said, beaming with delight.

``How so?'' replied Jean.

``Why, look you! I thought I was too when I heard the news. I am
perfectly delighted that you did not lodge with that pitiful fellow,
Loup. Between ourselves, I can't endure that man. He has recently---but
I will tell you about that another time. What was I saying? Oh, yes!
Look, there comes my cousin, the dear, good woman! You cannot imagine
how pious she is. His reverence, the Bishop, could tell you. Why, he has
even taken her confessions many a time himself!''

``Yes, but what does all this mean?''

``Why, it means good news. I have stolen away to tell you, for it is
still a secret, and my cousin has promised his reverence not to breathe
a word of it to any one, and she first told Charlotte---''

``But what is this secret?''

``Well, what do you think? The witch has actually had intercourse with
evil spirits in the prison!''

``Ah! How do you know that?''

``How? My cousin could tell you exactly. Let me see, how was it? Oh,
yes; I have it. The witch had promised to renounce all her hellish
practices and wear women's clothes. So they were brought into the
prison; but notwithstanding that she was found the next morning with
men's clothes on again. There you have it.''

``But why do you conclude from that that she has intercourse with evil
spirits?''

``Why? Do you still doubt? Holy Ursula! his reverence says so. My
cousin, the good woman, she could tell you all about it; but she has
gone just now to mass.''

``But you were going to tell me some good news.''

``Oh, yes; I had nearly forgotten it. It is this. As the witch has
resumed her intercourse with the evil spirits, she will have to be tried
again.''

``Well, of what interest is that to me?''

``Of what interest is it to you? Holy Ursula! Is it not of the greatest
interest to you that you have not come here in vain? When they sentence
the witch again, she will stand upon a high platform, as she did the
first time, and you will see her just as easily as you see me now.''

``So! That is nice. But when will it be?''

``I do not know, little friend. But, ps-t, my cousin will find out all
about it from his reverence.''

``Is the Bishop here?''

``Not yet; but if he does not come to-day, he will be here in the
morning.''

``Well, surely, I arrived here at just the right time.''

``Did I not tell you so? I am so glad you are not stopping with that
disagreeable Loup, for he could not have told you a word about this
matter.''

``Of course not. He has not such a pious cousin who confesses to his
reverence himself. But can I go now and see the tower and the cage?''

``Certainly, little friend; but listen. If you should meet that Loup, do
not greet him, do not even look at him, for they say he has an evil
eye---he might bewitch you.''

``I will keep it in mind.''

To his great disappointment Jean found the tower so well guarded that he
could not be of the slightest assistance to Joan. He decided to withdraw
and await events before forming any plans, and in the meantime make
inquiries about Marie. While on his way back he heard from passers-by
that the Bishop was momentarily expected, and that he would pass that
way. As he did not wish unnecessarily to expose himself to the prelate's
gaze, he entered the Ursuline church. It was empty. He went to the usual
spot, and scarcely had he placed his ear to the wall before he clearly
heard a sob, which seemed to come through the stone. Trembling with
excitement, he listened all the more intently, but in vain. All was
silent. Had he or had he not been deceived? All at once it seemed to him
as if it were the voice of the girl in the carriage which he met in the
forest, and that she could be no other than Marie of Chafleur. He
quickly made his plans. As he stood leaning against a door, near which
he had been kneeling apparently engaged in devotion, he pressed a piece
of wax against the lock, went at once to a locksmith's in an out of the
way street, and said his master wished a key made from the impression.

The next evening, when all Rouen was out to see the young King Henry of
England make his entrance, Jean again found the church empty. He tried
his key and it opened the door. He emerged into a long, dark passage-way
which skirted the wall. If he was right in his calculations, he would
find the prison between this passage and the church. He felt along the
wall, for he could see nothing. He was right. There was a door near the
corner. It must lead to the prison out of which had come the sound of
sobbing. With trembling hand he took another impression, groped his way
back, closed the door in the church wall, and departed. The next day he
obtained the second key. He now forsook the church for a time and
devoted his attention exclusively to the fate of Joan. The strangest
reports were circulated about her; but they were so incredible and
withal so dreadful, that he paid little attention to them. What pained
him the most was the certainty that he could do nothing to help her.

Thus matters stood on that 30th of May of the memorable year 1431. The
sun gayly shone that morning, and the birds sang joyously in the trees
and among the flowers. The doors of Rouen stood wide open. From far and
near the multitude gathered. There was a sea of heads on the sides of
the great market-place, and in the streets leading to it, and windows
and roof-tops were crowded. In the middle of the square were three high
platforms. Two of them, which faced each other, were evidently set apart
for those directly concerned in the proceedings. The general interest
centred, however, in the third platform, of the use of which there could
be no doubt. The flooring rested upon a pile of wood so arranged that
the logs made steps, and from the centre of the platform rose a stake to
a man's height. The base of the pile was surrounded with bundles of
fagots smeared with resin and pitch.

``Come on, little friend,'' said the innkeeper to Jean, as he went up
some stairs. ``I have a nice place for seeing. I am so glad you did not
stop with that miserable Loup---but, holy Ursula! are you ill? Your hand
is as cold as ice.''

``I am not feeling very well,'' replied Jean, ``and I would rather go
back again.''

``What! You don't mean to leave just as the spectacle begins! I will get
a little potion for you which my cousin, the good woman---but, holy
Ursula! the drums are already rattling. The judges are mounting the
great tribunal. Look, there is his reverence. He has the parchment in
his hand which contains the sentence. Pay attention. He will read it
soon.''

Jean did not hear a word. His eyes were fixed upon a distant spot
whence, accompanied by the roll of drums and the shouts of the
multitude, a procession was slowly making its way through the crowd.

``Do you see the cart?'' said the innkeeper. ``Do you see the witch in
it? She is sitting by the side of Father Martin. That holy man has been
praying by her side all night that the evil spirit may forsake her. Holy
Ursula! See how they have bound her! Her hands are fastened, and her
feet are in iron rings with a chain between.''

The cart soon reached the square. Joan was led up to the second platform
by Father Martin. The Bishop of Beauvais read the sentence amid the
profound silence of the multitude:

``In the name of God, Amen.

``We, the Bishop of Beauvais, Master and Vicar of the Inquisition,
pronounce sentence. As Joan, commonly called `the Maiden,' has relapsed
into heresy and apostasy, she is excommunicated, and herewith given over
to the secular power for the infliction of the punishment provided for
the heretic.''

Some would have applauded, but they found no encouragement, for Joan had
fallen upon her knees and was praying, and when she raised her head her
face was as the face of an angel. Many began to realize that she was not
a criminal, and loud sobs, indicating the growing change of feeling,
were heard here and there. Observing this, the judges hastened their
work. An attendant approached her and placed a pointed cap on her head
with the words, ``Heretic, relapser, apostate, idolatress,'' written
upon it. He then hurried her down the steps and led her to the pile, at
the foot of which the executioner was in waiting.

``Leave me not, Father Martin,'' she implored, as the executioner seized
her and dragged her up to the platform. The father followed and remained
with her as the executioner bound her to the stake and then turned to
descend.

``Pray for me, all pray for me,'' she cried to the people.

The executioner seized a torch and lit the fagots at the foot of the
pile. Swiftly rose the flames.

``For God's sake, my father,'' cried Joan, ``take care! Quick, quick,
hurry down, but hold the crucifix high before me until I die.''

Martin did as she requested. The Bishop of Beauvais approached.

``Bishop, Bishop,'' said Joan, reproachfully, ``you are the cause of my
death,'' and then as she felt the heat, she exclaimed, ``O Rouen, I fear
you will have to suffer for my death.''

The flames mounted higher. A dense cloud of smoke concealed her, but now
and then the wind swept it aside, and the people saw, not a devil's
witch, but a praying angel with marvellously beautiful eyes fixed upon
heaven. Suddenly the flames seized her garments. Her last word was
``Jesus''---then a piercing death cry, and all was ended.
