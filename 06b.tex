Thus perished the Maid of Orleans, rescuer of France. She died forgotten
and forsaken by him for whom she had done all, betrayed through the
greed of her own countrymen, accused from motives of revenge by her
enemies. She died the most cruel of deaths, and yet was as guileless and
pure as when she sat under the Fairy Tree tending her lambs. Joan is a
unique figure in the world's history. A simple peasant maiden, who could
neither read nor write, and knew only the Lord's Prayer, the Credo, and
the Ave Maria, she achieved such extraordinary results by her gift of
inspiration that her contemporaries and posterity in their efforts to
explain them have had to attribute so much of the miraculous to her
deeds that some have doubted her very existence.

The old market-place of Rouen now presented another spectacle. ``Alas!
alas! we have burned a saint,'' many said. The crowd remained a long
time, as if riveted to the spot, staring at the fire as it consumed the
last vestiges of the victim.

The innkeeper himself was so overcome that he forgot all about his
companion. When he turned to speak to him, Jean was gone.

\threeast
