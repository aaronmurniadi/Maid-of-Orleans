\chapter{The Coronation and the Capture}

\drop On the left bank of the Loire, a few miles above Orleans, is the
little city of Jargeau. At the time of which we write it was surrounded
by massive walls, and was considered a strong fortress. After the
raising of the siege of Orleans the weakness of their position was so
apparent that the Duke of Suffolk, with his brothers, Alexander and John
de la Pole, fell back to Jargeau. Within a few weeks the Maid of
Orleans, as Joan was now generally called, was before its walls with her
principal commanders. Her only desire now was to conduct the King to
Rheims for his coronation. Notwithstanding her wounds were not yet
healed, she left Orleans with La Hire, Dunois, the Duc d'Alençon, and
other officers, went to Tours, where the Dauphin was then stopping, and
requested him to follow her at once to Rheims. He was not disposed,
however, to grant her request. His kitchen, cellar, and money-chest were
once more replenished, and as life was now very enjoyable, he decided
that it would be rash to hazard such an undertaking until the way was
cleared; for notwithstanding the deliverance of Orleans, the enemy was
still holding the district. The leaders also declared that it would be
in violation of all the rules of war. They must first open the way, and
above all, Jargeau must be captured. Joan was obliged to submit to their
decision, and join them.

On the 20th of June---the day was Friday---the army arrived before
Jargeau. Learning that Falstaff was on his way from Paris with help for
Suffolk, no time was wasted. Preparations for the assault were instantly
begun, and on Saturday evening a breach had been made in the wall. Early
on Sunday morning Joan, in full armor, entered the tent of d'Alençon, he
being in chief command.

``Come, noble Duke,'' she cried, ``let us make the attack.''

``What,'' he replied, ``to-day? On Sunday?''

``Why not, noble sir? Obedience is the best service to God.''

``But, Joan, is the breach passable?''

``Undoubtedly. God has given the enemy into our hands.''

``But, in the meantime---''

``There is no meantime, noble sir. What fear you? Have you forgotten
that I promised to take you back safe to your wife?''

``Well, let the attack begin.''

``Forward, attack!'' cried Joan, as she left the tent, waving her
banner. The troops advanced, but the apprehensions of the Duke proved to
be well founded. The breach was too high. A ladder must be raised. In
the meantime, among the daring ones who had rushed forward, was Jean.
``Halt, boy,'' shouted La Hire, at the same time pulling him back. ``I
don't think your skull is tough enough to resist that fellow's club.''
He pointed to the breach. Jean looked up and saw a giant standing in the
opening, wielding a massive club, and laughing with fiendish glee as he
dashed everything about him to pieces. ``Wait a bit,'' said La Hire. ``I
think your namesake, the gunner, can stop that fellow's laughing.'' He
was right. The catapult hurled a rock through the air, the giant flung
up his arms, and fell backward from the wall, a great shout accompanying
his fall.

Joan rushed up the ladder shouting ``Forward, forward, my brave ones,''
but a stone felled her to the earth.

``Hurrah!'' shouted the English, ``the witch is dead.'' Their joy was
short-lived, however. Fear again seized them, as they not only heard her
assuring words to those about her, but saw her prepare to ascend the
ladder again.

``All right now, boy,'' said La Hire, as Jean again rushed forward. ``I
am with you this time.'' They quickly climbed the ladder, but when Jean
reached the top some of the English had been thrown down, and others
were flying into the city. Among those who had been hurled down was
Alexander de la Pole. When the Duke of Suffolk saw his brother fallen,
and the French pouring in, he gave up the fight, and, like the others,
turned toward the city.

``Halt! halt! surrender!'' a strong voice shouted. Suffolk stopped and
looked at his pursuer. He could have vanquished him with little effort,
but he did not consider it chivalrous to take advantage of an enemy.
``Who are you?'' he asked.

``Jean Renault,'' was the answer.

``Nobleman?''

``Yes.''

``Knight?''

``No.''

``Kneel down.''

Jean obeyed. The Duke raised his sword, and with the words, ``In the
name of God and Saint George I dub thee knight,'' he dealt him three
blows upon the shoulder with the flat of the blade, and then offered him
his sword.

Jean arose, pressed the Duke's hand to his lips, and took his sword,
saying, ``I do not deserve this honor, my Lord Duke, but I am very proud
to receive the sword of the first of England's heroes.''

``You are right,'' said a deep voice behind him; and, as if in
benediction, La Hire laid his mailed hand upon his head. ``You are
right, say I. All the knighthood of France would begrudge you this
sword. By Saint George, I am just as happy as if I had seen that sword
in the hand of my own son.''

``The noble La Hire's word,'' said the Duke, ``is sufficient warrant
that my sword will be worthily carried, Sir Jean Renault; there is no
stain upon it, guard its purity.'' Jean's feelings overcame him, and he
could make no reply.

After the capture of Jargeau, Joan rested for a time, meanwhile
forwarding reinforcements to Orleans, for more victories must yet be
achieved in the district of the Loire. While her fame attracted recruits
every day to her banner, the fear of her very name was so overpowering
that Meung, Beaugency, Guetin, and other cities surrendered without
offering resistance. The English force which came from Paris under
Talbot and Falstaff was defeated at Patay, and two of its generals were
taken prisoners. The evacuation of Paris was the result of this
battle.\footnote{It was after the victory at Patay that Joan of Arc
  declared that the English power in France would not recover from the
  blow in a thousand years.}

Joan returned with the Duc d'Alençon to Orleans, and thence repaired to
Gien to see the Dauphin. ``Sire,'' she said, ``the district of the Loire
is now clear. Go with me to your coronation at Rheims.''

The Dauphin still hesitated. ``The way is even yet dangerous,'' he said.
``Many castles and cities in Champagne are still in the hands of the
enemy. How easy it would be for them to fall upon our rear from
Normandy.'' His councillors in attendance decided that his fears were
well grounded.

``Oh, you saints of heaven,'' cried the Maiden, her eyes shining with
enthusiasm, ``help me to inspire the noble Dauphin with a little of that
courage you have given me!'' Her prayer was answered at once. The King
was moved by her soulful eyes, her steadfast faith, and her lofty
inspiration. ``Yes, Joan, we will trust you,'' he exclaimed. ``On to
Rheims.''

Orders were sent in all directions. The leaders and their troops quickly
assembled, and the march began. Joan led the vanguard. At the mere
announcement of her coming the cities of Auxerre, St.~Florentin,
Chalons, and Sept-Sceaux capitulated. Troyes did not surrender until
preparations for assault were made. At Sept-Sceaux, four leagues from
Rheims, they rested. Charles then sent three of his principal
councillors to San Remy to fetch the holy oil which was kept
there.\footnote{Tradition says that Clovis and all his successors for
  nine centuries were anointed with this oil.} They returned, escorted
by a grand procession headed by the Abbot of San Remy, who walked under
a canopy, carrying the phial.

From all the towers of Rheims the bells announced the memorable ceremony
of July 17, 1429, which completed Joan's mission. The pealing organ and
a majestic hymn of praise welcomed the long coronation procession as it
entered the Cathedral of St.~Denis. Joan accompanied the King to the
vestibule, where the Archbishop of Rheims met him and conducted him to
the high altar. The choir was occupied on each side by the commanders
and leading dignitaries, knights and lords, squires and attendants,
while a vast multitude of people crowded the cathedral to its utmost
capacity. Joan stood next to the King, her eyes shining with sacred joy,
holding her banner in her left hand and her sword in her
right.\footnote{Joan's enemies subsequently reproached her for this,
  saying it was pride that moved her to take her banner to the ceremony.
  She only replied that it had shared the pain; it was right it should
  share the honor.} It was a position which ordinarily only the first
marshals of the kingdom were entitled to occupy; but no one questioned
her right to it or envied her.
