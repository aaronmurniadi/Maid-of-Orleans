\chapter{The Dauphin and La Hire}

\drop By the storm of an April day in the year 1428, four years after
the events related in the preceding chapter, a man was detained at home
in the castle of Chinon.\footnote{Chinon, a town in the department of
  Indre-et-Loire, France, was a royal residence from the twelfth century
  to the reign of Henry the Fourth. In its great hall Charles the
  Seventh first saw Joan of Arc.} His costume showed that he was of the
highest rank, and the apartment also was furnished in a style of
princely luxury. As it was apparent, however, that these luxurious
surroundings were the survivals of an older period, evidently the
present occupant of the castle either did not care to improve them or
could not afford to do it. As a matter of fact the shabbiness of his own
costume favored the latter inference. The morose expression of his face,
which had but little that was attractive in it, deepened this
impression. Nothing about it indicated any higher ambition than the
gratification of his physical desires. His appearance gave the
impression that he was at least thirty-five years of age, but in reality
he was only twenty-six. This man was the Dauphin of France, afterwards
King Charles the Seventh.

Before him stood a young and beautiful woman, whose face was in striking
contrast with his. A dignified royal presence, eyes flashing with spirit
and resolution, womanly gentleness and kindness,---such were the
characteristics portrayed in her beautiful countenance. Some of its
lines indicated troubles of the heart, but her present trouble was of
another kind.

This lady was Marie of Anjou, the proud consort of the Dauphin. They
were both standing, for in the excitement of their conversation they had
evidently risen from their seats.

``Oh, this wretchedness!'' she moaned. ``Beautiful France desolate! The
luxurious fields of the Loire laid waste! The poor people killed or
fugitives in the forests! Townsmen in servitude or in continual fear of
a victorious enemy! And you! What are you doing?''

``How, I? Who laments all this more than I? Who has to suffer from it
more than I? Am I not growing poorer every day because of it? I am
afraid I shall not have even an ordinary dinner to-day.''

He hastily rang a silver bell, and a servant entered. ``Jacques, go to
the cook and ask him what he has for dinner.''

Contempt and deep sorrow were pictured on Marie's face, but she quickly
mastered her anger. ``Certainly, my husband, you have to suffer,'' said
she, ``but what kind of a king would he be who did not feel the
sufferings of his people a thousand-fold?''

``Pah! I feel my poverty above everything else.''

``But what necessity is there for your poverty? Do you not know that
your poverty will disappear on the day when you overcome the enemy?''

``I overcome the enemy! God help me! I need what few mercenaries I have
to forage for the kitchen. Raise fresh troops! How can I do it? What
little gold there is in my treasury is already pledged. I overcome the
enemy! Ha, ha, ha! they already hold nearly all of France. La
Hire\footnote{La Hire, one of Charles the Seventh's most distinguished
  generals, was born about 1390, and died at Montauban in 1443.} told me
to-day that Count Salisbury is before Orleans, and is besieging the
city. When Orleans falls, I must fly from here. Then what? I shall end
by being a beggar.''

``You will not if you pluck up courage and remember that when the
necessity is the greatest, divine aid is nearest, and that it is more
glorious and more worthy of a king to be vanquished in battle than to be
ruined by inglorious indolence.''

``Pah! I will do neither the one nor the other. I will live and enjoy
myself. That will compensate me for what I lacked in the hungry days of
my youth. I will make terms with the English. They may have everything
else if they will only leave me Languedoc.\footnote{One of the ancient
  governments in Southern France. Toulouse was its capital.} I can live
there in a manner that suits my income.''

``Shame! shame! what is this I hear?'' exclaimed Marie. ``Are you a
Valois? Does royal blood flow in your veins? Do you not blush to utter
such words? Oh, my husband, do whatever else you wish, but save France
and me from such shame.''

``Well, well, all is not over yet at Orleans.''

``But even if it fall, and if all seem lost, even then do not make such
a shameful agreement.''

With these words the noble lady retired. The moral indignation of her
manner and her words appeared to make some impression upon the Dauphin.
He buried his face in his hands, and was absorbed in thought---as far as
he was capable of thinking.

While thus engaged, an arras door opened behind him, revealing the
charming little curly head of a girl of eighteen or nineteen years. No
one could have seen that exquisite figure moving along with such easy
and consummate grace without confessing he had never before seen such
exquisite beauty and fascinating manner. Her charm appeared not only in
her beautiful figure, but also in the gracious expression which
characterized her personality and radiated from her countenance.

This maiden was the famous Agnes Sorel,\footnote{Agnes Sorel was born in
  Touraine about 1409, and died in 1450.} the favorite of Charles the
Seventh, who, as history relates, was conspicuous for her womanly
tenderness, and who always used her influence over the King for noble
purposes and never for personal ends.

The Dauphin was not aware of her presence until he felt the light touch
of her hand upon his shoulder. The sight of her was magical in its
effect. His face lightened up, and all traces of dejection disappeared.

``Is it you, Agnes? Now everything is all right.''

``What has been wrong?'' she asked most tenderly.

``Marie has been here. She has made my head ache and has nearly ruined
my appetite. But---''

``I know all about it,'' interrupted Agnes.

``How? You know all about it? Who could have told you?''

``No one told me.''

``Oh, you have been eavesdropping. Ah, ha!''

``I had to. I could not go back, and of course I was not permitted to
enter.''

``Hm! Never mind. It is all right, just the same.''

``Oh, no, Your Majesty.''

``How? What do you mean?''

``I share the anxiety and trouble of your proud consort.''

``Nonsense! You ought not to be troubled.''

``By all the saints, Your Majesty, I shall be inconsolable and unhappy
if you do not abandon your decision. I should be ashamed to serve a
prince who can so easily renounce his rights and his dignities.''

``Well, well, I will consider the matter. Will that satisfy you?''

``Oh, no, sire. You must promise me that you will not think again of
that hateful scheme. Will you not for my sake?'' Thereupon she
triumphantly and gracefully pirouetted about the apartment.

``Agnes, I take back my word,'' cried the Dauphin.

This made her all the happier, and she continued her dance, singing this
accompaniment:---

``Eio, eio, eio, no, He cannot be a King Who does not keep his word!
Eio, eio, eio, O, This one here---he is not such, No, no, no, oh, no.''

With the last word she suddenly disappeared, for the heavy tramp of
men's feet was heard in the antechamber. The interruption displeased the
Dauphin, and he was about to leave the room, but before he could do so
the new-comers stood at the door. It only increased his displeasure that
he was forced to remain. The two men, whom he regarded with a sinister
expression, were rough and sturdy, men of the class who stand fast in
battle and look death fearlessly in the eye, knights in the truest sense
of the word.

``So quickly back, my brave La Hire?'' said Charles to one of them.

``By Our Lady, Your Majesty, never was there greater need for quick and
decisive action than now,'' was his reply. ``I have just heard that
Count Salisbury has completely invested the city of Orleans. Not even a
cat can get out of it, and in a few weeks it will be in the clutches of
famine. If we do not help them you can easily see---''

``Help them!'' interrupted the Dauphin, despondently. ``My good knight,
how much money do you suppose there is in my treasury? Ha! ha!''

``The people will see to it that the treasury of their legitimate King
is filled if in turn they have the assurance that he will make a stand
for the right, for his honor, and for the fatherland.''

``And until then I suppose I can keep on with my fasting cure to which
my mother accustomed me. You will not believe it, my good La Hire, but
it is the sad truth that my cook has notified me he has nothing to serve
to-day but a pair of fowls and a hind-quarter of mutton. And you are to
be invited as guests to such a banquet as that!''

``Well, sire, that is all right. To-day we will eat the fowls and the
mutton; to-morrow we will drive the English out of their kitchens, and
seat ourselves at their tables.''

``But how are we going to drive them out? It is impossible. Can I summon
troops out of the ground?''

``Yes, sire, you can!''

The Dauphin looked at him with astonishment.

``Do you take me for a wizard? Or, do you mean I am in partnership with
the devil?''

``Resolution and courage, sire, have often worked wonders. Inscribe them
on your banner to-day, and to-morrow it will not flutter deserted. It
will rally those around it who have fallen away discouraged as well as
those who follow the profession of arms, and would gladly enlist under
such a royal banner for the sake of the rich reward. There are men yet
who are ready to stand by you with their good swords. See, here is my
stanch friend Saintrailles,'' pointing to his companion, ``and he is not
the only one who is ready.''

``You are welcome, brave knight,'' said Charles. ``It is a shame I can
only invite you to sit down to two fowls and a leg of mutton.''

``Sire,'' replied Saintrailles, who could hardly restrain his
indignation, ``I was not thinking of your table when I followed my
friend here. I was thinking of your wretched plight and of the bleeding
fatherland.''

``And do you believe it can be helped?''

``Certainly, sire, but he who would win must venture.''

``Yes, and in the meantime he may also lose. But, by my faith, I have
not much more to lose.''

``But all the more to win. The brave soul thinks only of winning.''

``Oh, yes, you talk like La Hire, and La Hire talks like Marie, and
Marie talks like---but if the English would let me have Languedoc as an
independent dukedom, then---''

He did not finish the sentence, for through the side-door, which was
partly open, he saw the warning finger of Agnes Sorel. Then he resumed:

``I am glad you have come, noble knights. We will meet at table and
further consider this matter. But, alas! two fowls and a leg of
mutton!''

On the evening of the same day, when La Hire reached his lodgings and
was laying off his armor, a young man of about eighteen years entered.
His strong, supple frame, handsome, noble face, piercing black eyes,
lofty forehead beneath raven-black hair, as well as his resolute,
self-confident bearing, impressed themselves upon the knight.

``Who are you, and what do you wish?'' he said, at the same time
regarding the young man with evident satisfaction.

``My name, noble knight, is probably unknown to you,'' was his reply.
``My father of blessed memory, however, left it to me unstained. I have
come to honor that name under your banner in the service of the
distressed King and the unhappy fatherland.''

``Well said, young man, and, by Our Lady, you look to me like one who
can use his sword as well as his tongue. We will consider the matter.''

``Will you not accept my service, noble sir?''

``Gently, young man. Do you suppose that I confide the honor of my
banner to every nameless fellow? Out with your name.''

``I am called Jean Renault.''

``Renault? Was your father that Thomas Renault who fell in the service
of the Duke of Orleans, fighting against the English?''

``The same, noble sir.''

``Then a thousand times welcome. Your father was a brave knight and a
noble gentleman. From to-day you shall serve under my banner, and you
will have ample opportunities to earn your knightly spurs.'' Thereupon
he shook the young man's hand heartily.

``I thank you for your confidence, noble sir,'' replied the new
adherent, with beaming eyes. ``I will do my utmost to justify this
confidence, but what I can do to earn my knightly spurs I do not yet
know, partly because of my youth, and also, though it is no disgrace,
partly because of my poverty.''

``Poverty! Your father had property.''

``Yes; but it was at Rouen, and it has fallen into the hands of the
English.''

``Well, we will see that it is returned to you. But now tell me where
you acquired your training.''

``Under my father, to whose retinue I was last attached.''

``Then you have also fought against the English?''

``Yes, I was in the battle in which my father fell and the Duke of
Orleans was captured.''

``Then you are doubly welcome, my young friend,'' warmly exclaimed the
knight. ``I well know I cannot take your father's place, but I will do
for you all that a man can.''

Overcome by such generosity, Jean pressed the knight's proffered hand to
his lips. His heart was too full for words. La Hire understood his
silence, and admired him all the more. ``You are from the neighborhood
of Rouen, and are acquainted there?'' he resumed.

``I know every village thereabouts, noble sir. Alas! they are nearly all
ruined.''

``Yes! God and the saints pity them. But, further, do you know the
Bishop of Beauvais?''

``Certainly I know him. It is his diocese.''

``That is fortunate. I have a message for the bishop, but no messenger
who is acquainted with that region, or cunning enough to evade the
English. I can trust you for both?''

``I am ready, noble sir, provided you do not wish me to act as a spy.''

``Do you suppose, my young friend, that I would choose you if I needed a
spy? No, the mission you are to undertake has nothing to do with the
war. However, I cannot conceal from you the danger involved in the
undertaking. The Bishop of Beauvais has the reputation of loving money
and leaning to both sides. Do you understand me?''

``Perfectly, noble sir. He is devoted, now to the Burgundians, now to
the Lotharingians, now to the English, and now to the Duke of Orleans.''

``Listen. The English might easily regard a messenger to him as a spy,
which, by Our Lady, would grieve me. But then again, even if they should
hold you as a prisoner it would be uncomfortable, for money is so scarce
in our treasury that you might have to wait a long time for your
release.''

``I do not think, noble sir, that the English will catch me.''

``Then you will undertake the mission?''

``I await your commands.''

``Rest to-day and to-morrow. The day after to-morrow you shall have the
letter for the Bishop.''

As the road to Rouen led directly through the English district it was
practically impossible for a messenger to make the journey on horseback.
Jean therefore decided to go on foot, disguised as a peasant. As the
cities around Orleans were in possession of the English, he was
continually forced to take divergent routes. He made a wide circuit
around Paris, and at last approached Rouen from the east. While on this
part of his journey he stopped in a forest one noon to rest and enjoy
his simple meal. While thus engaged, he suddenly heard a female voice
crying for help. He sprang up, and ran to the road whence the cry had
come. Concealing himself behind some bushes, he watched and listened. He
heard the distant rattle of a carriage and the clatter of armor toward
the east. A heavy travelling carriage soon came lumbering along the
rough road, accompanied by half a dozen men at arms.

``Has a shameful crime been committed, and did the cry come from that
carriage?'' said Jean to himself. ``What! I think I know the arms on the
carriage door. Why, certainly. They are the Duke of Luxemburg's. But I
must be sure of it.'' With this he rushed from his hiding-place.
``Halt!'' he shouted, brandishing his knobbed stick.

The coachman and attendants were astonished. It seemed incredible that a
single man, armed with such a weapon, should dare to order them to halt.
While they prepared for resistance they watched, not so much the young
man as the thickets, for they were suspicious that other peasants might
make their appearance. During this brief waiting Jean discovered what he
had feared, and what he was so anxious to ascertain. Scarcely had his
``halt'' died away when a girl's face appeared at the carriage door.

``Help! help!'' she cried, in terror. ``Help! They are dragging me to a
convent---''

A smothered exclamation of pain followed the last word. Some one inside
the carriage had pulled her back and stifled her cries. Instead of the
girl's face there now appeared at the door the wrathful face of a
knight.

``Seize the dog,'' he shouted. ``Do not kill him. I must have him
alive.''

The men at arms prepared for action at once, but Jean did not stir. He
stood immovable as a statue, staring at the door. The distress which he
was powerless to relieve threatened his own undoing, but he remained as
if glued to the spot, trying to identify the personality of the victim.
He had only caught a fleeting glance of her, but that glance left an
impression that could not be effaced. She was a girl of fifteen or
sixteen years, and so radiantly beautiful that even her expression of
poignant suffering and fear could not diminish her charm.

Meanwhile the men at arms were arranging their plan. They evidently
intended to surround and overpower him, but their movements were too
slow to suit the knight in the carriage. ``Well,'' he roared, ``what are
you waiting for? Seize him!''

The command brought Jean to his senses, and the first glance revealed
his danger. With a quick rush he broke through the circle of his
assailants and ran back into the thicket.

``Follow him, ride him down,'' furiously cried the knight.

The men at arms rode after him, but before they could overtake him he
had disappeared in the woods, where they could not follow him on
horseback. To dismount and pursue him on foot would have been a rash
undertaking, so they turned about only to receive violent reproaches and
curses from their master, who was forced to resume his journey without
his wished-for victim.

Jean did not go far, for he well knew they would not dare to follow him
into the forest. Leaning against a tree, he watched the carriage, which
took the road to Rouen. His first impulse was to follow it and keep it
in sight, but, upon second thought, he remembered he was not at that
moment his own master, but was in the service of another, and that under
such circumstances he had no right to risk his liberty or his life.
Accordingly he let the carriage go on several hours before he resumed
his journey.

Making allowance for the precautions he must take, it would be three or
four days before he could reach Rouen. On the way he made several
inquiries as to the whereabouts of the carriage, so that when he entered
that city on the evening of the fourth day, he knew it was there. At the
inn where he put up he passed himself off as a fugitive peasant who
desired an interview with the bishop, that he might tell him of the
sufferings of himself and his fellow villagers. As his story was a
probable one, he hoped there would be no opposition to his remaining
there. He was told that the bishop arrived two days before in the
company of the Duke of Luxemburg, and had brought a young novice to the
convent of Saint Ursula. He had gone away again with the Duke, but only
for a short time.

Whenever Jean ventured out of the inn, he took his way to the convent.
He could see only its outer walls, and yet he was drawn to it over and
over again. Near the convent stands the church of Saint Ursula. As its
doors were always open there was nothing to prevent him from entering
and praying fervently for the unhappy girl he had seen in the forest.
One day he as usual selected a spot close to the wall between the church
and the convent for his devotions. This wall must have been in frequent
use, for there was a door in it opening upon a passage-way to the other
buildings. While Jean was praying the church was empty, and in the
gathering shades of evening the sacred room was quiet and restful. In
the profound silence it seemed to him that he heard human sobs in the
distance. He listened intently. There could be no doubt of it. He was
not deceived. The sound seemed to come out of the wall. He placed his
ear against the stone, and distinctly heard a woman's painful
ejaculations between alternate groans and gentle sobs. A cold sweat
stood on his brow. He felt rooted to the spot. The longer he listened
the fiercer grew the storm in his breast. At last he could endure it no
longer. He rushed out into the air. His heart was almost bursting. ``The
captive lady!'' he cried, ``can it be she?''

His despair drew him again to the spot, and again he listened. His pulse
beat so feverishly, and he was under such excitement, that it was
impossible for him to judge calmly, but he fancied he recognized the
voice.

During the remainder of his stay in Rouen Jean spent his time almost
exclusively in trying to discover the fate of this unfortunate one, but
it was in vain. He only found that he was drawing attention to himself,
and this attention at last became so apparent that after delivering the
letter to the Bishop he was forced to leave Rouen abruptly and make his
way back.

\threeast
