The Bishop's eyes glistened. ``And the security?'' he said, stretching
out his hand.

``My word, the word of a nobleman;'' and they shook hands.

A pause ensued. Each of the men, in the stillness, seemed to be studying
whether he might not find eventually that he had been overreached and
had not received his proper share. The Bishop was the first to come to a
decision, and asked, ``When shall we begin our work, noble friend?''

``At once, if you are ready.'' Thereupon he rang a bell, and ordered the
servant who answered it to call Mademoiselle de Chafleur.

It was not long before Marie came running into the room, full of joyous
exultation. ``Dear uncle, see these beautiful violets,'' she cried.
``Oh, what delicious perfume!''

``Very beautiful indeed. They are messengers sent by Spring to the other
flowers.''

``It must be so. Oh, you cannot imagine how beautiful the park is
already! Tell me quickly what I am to do, so that I can return soon.''

``So you find it very pleasant in the park?''

``Oh, I could stay there always.''

``I am all the more sorry, then, that you will have to leave it soon.''

``What! Leave! Uncle, I do not understand you.''

``Yes, child. The tumult of war approaches nearer and nearer.''

``What of that? Is not the castle safe? Let the Englishmen come. We will
send those long-nosed gentlemen home again. Yes, `we,' I say, for you
know I am a Chafleur.''

``I have the highest respect for your courage, my little Amazon, but the
English will not be greatly scared by it. No, child, I must find a safer
place for you.''

``And my aunts?''

``Oh, that is a different matter. My wife and sister must submit to the
inevitable.''

``And I can also.''

``No, child. Your father sacredly intrusted you to me. I should not be
keeping my word if I exposed you to the dangers of war.''

``But I say again, uncle, and you have said yourself, that the castle is
safe enough.''

``Still it can be taken; but no enemy will dare to attack the sacred
walls of a convent.''

``A convent! What do you mean? Do you intend to make me a nun? Me! A
nun! Ha! ha! ha! I shall die a-laughing.''

``It is not always nuns who find shelter in a convent.''

``Nevertheless, uncle, and once for all, I say I will have nothing to do
with a convent.''

``Then tell me what you will do, for you cannot stay here.''

``Are you in earnest, uncle?''

``Absolutely so.''

The tears came to the girl's eyes. Sobbing, and throwing her arms around
his neck, she exclaimed: ``Uncle, you cannot send me away from you.''

``It is for your safety, my child.''

``But I do not wish any special safety. Where my aunts can stay, I can
stay.''

``It is of no use. No use. My decision is final.''

The girl stood erect. She wiped the tears from her eyes and looked at
the knight with a strange and distressed expression. Gradually her look
became colder and more fixed, and at last he realized her undaunted
determination.

``My decision is made too, uncle. I will not go to a convent. I would
rather fall into the hands of the English. But the situation is not so
desperate as that. I will let my kinsman La Hire know. He will protect
me. Let me have a messenger, uncle. In an hour I will have a letter
ready.'' Thereupon she left the room.

``Well, what do you think now, noble knight?'' began the Bishop.

``Pah!'' he replied, ``I will send her a messenger who will throw her
letter into the first forest brook he comes to, and return without
seeing La Hire.''

On the morning of the fourteenth day after this scene, a heavy
travelling carriage stood in the castle yard with an escort of six armed
men. Marie lay sobbing in the arms of Madame de Luxemburg. Still
sobbing, she at last followed the impatient lord of the castle to the
carriage. Nothing had been heard from La Hire, and when, as John of
Luxemburg had said, an attack upon the castle was likely to be made, he
told Marie he would accompany her to her kinsman. At the first inn they
met the Bishop of Beauvais, apparently by accident. As he was journeying
in the same direction he accepted the knight's invitation to take a seat
in the carriage.

Overcome with grief, and not expecting any trickery, Marie at first did
not notice the road they were taking. After passing three or four inns,
however, she saw that they were going west instead of south. Not even
then did she suspect treachery. They easily satisfied her inquiries by
pretending they must take a circuitous route to avoid encountering the
English. When, however, they kept on in the same direction the next day,
her suspicions were fully aroused.

``Uncle,'' she said, ``you cannot deceive me any longer; you are not
taking me to Chinon. What are you going to do with me?''

``I will not deceive you, child,'' replied the knight, for pretense was
useless any longer. ``I cannot carry out my plan to take you to Chinon.
The whole district of the Loire is in the hands of the English. I cannot
even get back to Beaurevoir, so nothing remains but---''

``But what?'' she piteously exclaimed.

``The convent.''

She uttered a scream of terror.

``Be quiet,'' said the knight, harshly. ``If you scream again I will
silence you in a way that may not be agreeable.''

They were in a forest where fugitive peasants might be in hiding. Even
at a distance from it, he had been fearful lest the girl might attract
some one's attention. He wished to reach his destination without being
observed, and was particularly anxious no one should even suspect where
he was or what he was doing.

Marie was not frightened by his threat, but a quick glance showed her
they were in a forest where no help of any kind could be expected. In
despair she sank back into a corner of the carriage. Anger, desperation,
and scorn raged by turns in her breast, until at last, overcome by
exhaustion, she buried her face in her hands and wept.

The vigorous ``halt'' of a manly voice aroused her from her wretched
condition. In an instant she was at the carriage door. Her first glance
fell upon a handsome youth who was advancing courageously toward the
carriage. The reader knows who he was.

``Help! help!'' she involuntarily cried. ``They are taking me to a
convent.''

Her guardian pulled her back, and silenced her cries by holding his
handkerchief over her mouth. She tried desperately to release
herself,---but what availed her weakness against the strength of a
trained knight? In her anguish the image of the brave youth rose before
her, and her anxiety about his fate made her forget her own. She
listened intently to all that was going on outside. She trembled when it
seemed impossible for her to escape, but at last she exulted when she
knew that he was safe.

It was late at night when the carriage came to a stop. Marie knew by the
call of a watchman that they were either before a city or a castle. The
Bishop gave his name, and the creaking gate opened. The carriage passed
through several dark streets, and stopped at last before a large, gloomy
building. Here also the Bishop's name was an Open Sesame; the heavy
bolts were pushed back, the carriage rolled over a paved yard, and with
a hollow, fateful sound the gate was closed and locked.

Marie shook as in an ague fit. She realized that she was a prisoner, and
perhaps was cut off from all the pleasures of life; but not a sound
escaped her lips. Her mute sorrow alone reproached her persecutors. She
did not know she was in the Ursuline Convent at Rouen, but she had no
doubt it was some convent in the Bishop's diocese. Evidently they were
ready to receive an exalted guest, whom they had expected, in a manner
befitting her station. The abbess, a lady of middle age, who, judging by
her speech and manners, might have been of high rank, was awaiting her
in the parlor. After the Bishop had exchanged a few words with her, the
abbess turned to Marie and said: ``May your entrance among us be blest,
Mademoiselle de Chafleur. I hope these sacred walls will furnish you
both the outward security which you need, and your heart that peace
which the world cannot give.''

There was something so cordial, and withal so winning, in the tone with
which she spoke these words, that Marie pressed her extended hand to her
lips with the utmost sincerity, and covered it with kisses. She longed
to throw herself into the arms of this gracious lady, and weep away her
sorrow as she would on a mother's breast. Her longing was so
overpowering that she sank upon her knees and moistened the abbess's
hand with her tears.

``Save me, gracious lady, save me,'' she implored. ``I am the victim of
a conspiracy. They have deceived me, brought me here by force, and torn
me from all that is dear and sacred to me.''

The astonished abbess cast an inquiring glance at the Bishop. ``The
novice,'' he said in reply to it, ``is here because it is the wish of
her guardian, a lord of Luxemburg, who alone has authority to act for
her. Therefore it is idle to talk of force. To your---''

``I am not a novice,'' cried Marie, rising. ``I am Marie of Chafleur. My
guardian has control of my property, but he has no right arbitrarily to
dispose of my person.''

``I trust your ability,'' resumed the Bishop, ``to remove these worldly
ideas, which are unbecoming within these sacred walls, and to implant in
this perverse soul the spirit of quiet resignation and Christian
humility. I authorize you to employ all the means which are at your
command to produce this result, and I have no doubt of their efficacy.''

The last words were spoken with a peculiar intonation which was in the
nature of a command to the abbess, but of the significance of which the
poor child had not the most remote idea. The abbess, who understood well
enough what was expected from her, made a quiet sign of assent, and the
two men took their leave, firmly convinced that their work was completed
successfully.

Marie was assigned to the usual cell and left alone. She first went to
the grated window. It looked out only upon the yard. With a pitiful sob
she threw herself upon the hard couch. Her tears flowed, and she gave
vent to her anguish in melancholy ejaculations. At last she knelt before
the crucifix and poured out her aching heart in long and fervent prayer.
Again she quietly sought her couch. She was now able to think calmly
over recent events. As she was ignorant of what was in store for her,
she was still buoyant with the hopefulness of youth. She thought of La
Hire, whom she had known as an honorable knight. The image of the young
man also mingled pleasantly in her thoughts of the future. She decided
she would write again to La Hire. He could not have deserted her. Thus
consoling herself, she sank into kindly slumber. Poor child! Little she
knew that her letters could not find their way into the outside world
without first being read by the superior.

One day two nuns, commissioned to acquaint her with the rules of the
Ursuline order, visited her. Her declaration that she did not wish to
know them made no impression upon the sisters. They performed their
duty, and then withdrew to make their report. Shortly afterwards another
sister entered, and summoned the novice to prepare herself by prayer and
fasting for the vow which she was shortly to take.

``What means this farce?'' said Marie. ``I am not a novice. I will not
join your order. I will not take a vow.''

``Our wishes are useless within these walls,'' replied the sister. ``We
must do what the superior, the abbess, and the rules of the order
command.''

``What is that to me? I am not one of you.''

``You will do well, sister, to submit to the inevitable.''

``And what if I do not?''

``Then they will force you to submit.''

``Force me, Marie of Chafleur! I should like to hear how they propose to
do it.''

``I can tell you, sister. They will lock you in your cell and let you go
half starved.''

``Well, I would rather wholly starve than take the vow.''

``They will thrust you into a gloomy prison.''

``Go on.''

``They will come daily to your prison and punish you without mercy.''

Marie shrieked aloud. She clenched her fists. Her lips quivered.
``Woman,'' she at last exclaimed, ``the devil has sent you to tempt me!
Leave me. Go and report that I will suffer death rather than consent.''

``I must first do what I have been ordered, sister.'' Thereupon the nun
knelt before the crucifix and repeated aloud the prayers which were
prescribed as a preparation for the vow. When she had finished she
withdrew. What she had said came to pass. Marie first was locked in her
cell and given only a scanty bit of bread. When that proved of no avail
she was put into the prison. It was her loud laments which Jean had
heard while praying in the church of Saint Ursula, for the prison was
only separated from the church by a single wall.

\threeast
