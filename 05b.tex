The sacred function began at nine o'clock in the morning and lasted
until two o'clock in the afternoon. The opening ceremony was the
administering of the oath by the Archbishop, during which Joan,
following the old custom, held her sword over the King's head. Then
followed the knighting, for Charles had not yet received this honor,
without which he could not ascend the throne. He knelt and the Duc
d'Alençon knighted him. The third ceremony was the consecration and
anointing with the holy oil, and was performed by the Archbishop. The
last act was his coronation by the same prelate. As soon as the royal
symbol glittered upon his head the cathedral resounded with the
enthusiastic acclamations of the great multitude: ``Hail, hail, King
Charles the Seventh!'' accompanied by a fanfare of trumpets, the roll of
drums, and majestic chorales.

Joan was the first to proclaim allegiance to the Crown. She threw
herself at the King's feet, and after kissing his knee, said: ``Sire,
the will of God is accomplished. You are now the true King of France. My
mission is ended. Permit me to return to my home and resume the humble
life of the shepherdess.''

``No, Joan,'' replied the King, ``I cannot spare you. All that I now am
is due to you. You must accompany me on the return.''

Joan rose sadly. She felt that in remaining longer she would be
disobeying the divine voices which had commissioned her to perform only
the two tasks now successfully accomplished. The King rewarded her by
granting a patent of nobility to her whole family, whence it is that she
is called ``Jeanne d'Arc.'' Her coat of arms was a blue shield with two
gold lilies and a silver sword bearing a golden crown on its point.
These distinctions, however, were of little interest to Joan. She grew
sadder and sadder, and ardently longed for her home fields and her loved
Fairy Tree. This feeling became all the more intense when her brother
Pierre arrived; but she rushed joyously into his arms and was somewhat
consoled when the King appointed him her page, and she knew that he
would never leave her. She took part in many more military operations;
but although she entered many cities whose gates opened at the sound of
her name, though she was everywhere greeted as a saint and welcomed with
enthusiastic acclamations and songs of praise, she no longer felt the
early unquestioning faith and the sacred inspiration. An ill-starred
movement against Paris, in which she was wounded afresh, confirmed her
in the belief that she had exceeded her duty, and that she was no longer
under the protection of her saints. She was haunted with gloomy
presentiments of death. They pursued her in dreams, and at last she
again implored the King to let her go.

``What do you fear, Joan?'' said the King. ``If you are wounded it shall
be my care to heal you. If you are captured by the English I will
release you, if it costs half my kingdom. You are the guardian angel of
France. I cannot let you go.'' He placed her in command of his own corps
and sent her once more into the tumult of battle.

On the 27th of May Joan appeared with her army before
Compiègne,\footnote{Compiègne, a town in the department of Oise,
  forty-five miles northeast of Paris, and famous as a royal residence.
  Its palace was rebuilt by Louis XV., and fitted up sumptuously by
  Napoleon I.} which was occupied by the French but was closely invested
by the Duke of Burgundy, who was in alliance with the English. She
successfully entered the city, and of course was received with the
greatest enthusiasm. Early on the next day she made a sally at the head
of six hundred troopers. She wore her usual armor, with a short
silver-gilt cape over it, and carried her small battle-axe, sword, and
banner.

Philip of Burgundy's army was composed of experienced troops, and its
various divisions were led by Noyelles, John of Luxemburg, and John of
Montgomery. Joan swept among them like a whirlwind, carrying everything
before her, and for the time throwing them into utter confusion. A cry
of terror---``The Maiden, the Maiden''---was raised in the camp, but
when Philip of Burgundy appeared with reinforcements the English,
recovering from their first surprise and confusion, began to hold their
ground. Finding herself confronted by a tenfold increased force, she
ordered a retreat. She was the last in the line, and was closely pressed
by the enemy, but when the boldest of them came too close she turned
upon them and drove them back. In this manner she forced her way
successfully to the gate. As there was much crowding and disorder there,
she turned once more at the head of her rear guard against her pursuers,
and beat them back, thus gaining time for her troopers to get into the
city; but when she herself made a dash for the gate she found an English
troop barring the way. She slashed right and left and hewed her way
through; but, alas! the gate was shut. No one heard her call, no one
opened the gate, for it was feared that the English might rush through.
Joan turned her horse, hoping to reach open country or find another
gate. The enemy, seeing that she rode alone, plucked up courage. She was
quickly surrounded, and a desperate fight ensued. An archer stole under
her horse, seized her by her velvet cape, and pulled her down. She
gathered all her strength for a last effort, but, overcome by superior
numbers, sank exhausted upon her knee, and still fought on with her
little remaining strength. Longingly she watched the city, but no one
came to her rescue. At last she surrendered her sword to Lionel, one of
the leaders in the Duke of Luxemburg's corps.

``The Maiden is captured,'' shouted the soldiers. The news flew from
place to place and from troop to troop. The English celebrated the event
with as much enthusiasm as if they had won a pitched battle. Well might
they rejoice, for Joan's prowess had cost them two-thirds of their
French possessions.

The Duke of Bedford, the Earl of Warwick, and the Bishop of Winchester
instructed Brother Martin, Vicar General of the Inquisition, to demand
the delivery of ``the witch'' into the hands of the Church. Martin wrote
to the Duke of Burgundy as follows:---

\begin{quote}
“By virtue of the regulations of our order and of the Holy Roman See
which give us the authority, we entreat and command, under lawful
penalties, that you deliver to us the prisoner, the Maiden Joan, who
is suspected of heresy, that she may be proceeded against in the Court
of the Holy Inquisition.”
\end{quote}

``Both of us know,'' said the Duke of Burgundy to John of Luxemburg,
``that Joan is not a witch but a noble maiden, and that we are bound to
deliver all noble prisoners to our English allies for a consideration of
ten thousand pounds. But we also know that the Maiden is an exception,
as it is altogether probable that Charles VII will ransom her, for he
has promised to do so.'' John of Luxemburg was satisfied, as he hoped to
get more from the King than from the English. In the meantime he sent
Joan to his castle Beaurevoir, where she was affectionately greeted by
his wife.

Month after month passed, but nothing was heard from Charles VII. In the
luxurious life he was leading he had not time to think of his rescuer,
whom he had promised to ransom even if it cost him half his kingdom. For
this reason the English were anxious to expedite matters. They
instructed Cauchon, Bishop of Beauvais, in whose diocese Joan had been
captured, to request her delivery to him and to conduct her examination.
They offered ten thousand pounds to the Duke of Luxemburg, the ransom
price of a general, and an annuity of three hundred pounds to Duke
Lionel. In the middle of September the Duke of Luxemburg sent word to
his wife he could wait no longer, but by her earnest pleading and by her
excuse that the Duke of Bedford had not yet sent the money, she secured
a further respite for Joan.

Joan burst into tears as she now for the first time realized the actual
character of the situation. ``Oh, I knew it would be so!'' she
exclaimed. ``They have sold me, but I would rather die than be given
into the hands of the English.''

One stormy November evening the castle guards heard a scream which was
audible even above the howling of the gale. They rushed to the spot and
found Joan in the moat. She had thrown herself from her window, but she
had failed in her purpose. She was not dead. The event made the
avaricious master of the castle fearful that he might lose his reward
entirely, for how could he give security that this desperate maiden, in
spite of the utmost watchfulness, might not carry out her purpose yet?

A few weeks later the rabble of Rouen stood before an iron cage
suspended from a tower. Derisive epithets and cruel insults passed from
lip to lip and were greeted with indecent laughter. In a corner of the
cage sat a cowering figure bound with fetters. Her face could not be
seen, for her head was bowed in anguish. One of the mob thrust his lance
toward her to make her look up. He was successful. She slowly raised her
head, and the crowd looked upon eyes full of sorrow, eyes full of purity
and beauty,---the eyes of Joan. The Duke of Luxemburg had completed his
infamous bargain. He had delivered her to the English.

\threeast
