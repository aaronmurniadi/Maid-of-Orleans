\chapter{The Fairy Tree}
\setcounter{mopagecnt}{1}

\drop As the traveller, descending the valley from Neufchâteau,
approaches the village of Domremy,\footnote{Neufchâteau and Domremy are
  both in the department of Vosges, France. The former is a town with
  about 4000 population; the latter, a village, famous as the birthplace
  of Joan of Arc.} he will observe at his right upon an eminence of the
nearest range of hills a stately chestnut-tree, its lower branches hung
with wreaths of flowers, some fresh, some fading. If he does not mind a
little fatigue and climbs to this spot, he will be richly rewarded for
his exertions. The tree in itself is a sufficient compensation for his
efforts, for who does not contemplate with admiration such a work of
nature? Who does not listen with rapture to the gentle rustle of its
leaves and find rest in its cool shade? But this tree has a still
stronger attraction for those who believe its story. Ofttimes in the
twilight they see happy sprites dancing round it with joyous faces, and
the soft rustling of its leaves they declare is celestial whispers, for
it is given to them to understand heavenly speech.

This tree is the ``Fairy Tree.''\footnote{One of the witnesses at the
  trial of Joan of Arc said: ``There is a tree called by us the `Fairy
  Tree.' Every year the young girls and youths of Domremy come to walk
  there on the Lætare Sunday. Jeanne the Maid went there like all the
  other girls, and did as they did. Though she hung garlands on the
  boughs of the `Fairy Tree,' she liked better to take them into the
  parish church and lay them on the altars of Saint Margaret and Saint
  Catherine.''}

The outlook from this spot will still further repay the traveller. A
beautiful valley spreads out before him, bounded on either side by the
forest-crowned heights of Argonne and Ardennes, between which the
Meuse\footnote{The river Meuse flows through France, Belgium, and the
  Netherlands, a distance of 500 miles, and empties into the North Sea.}
winds its silvery way. Numerous villages dot these heights and are
sprinkled here and there along the lower pasture-land. North and south
gleam the towers of Neufchâteau and Vaucouleurs.\footnote{Vaucouleurs is
  a town of about 3000 population. It was from there Joan of Arc started
  on her expedition to save France.} The nearest, and at the same time
most pleasant of these villages, is Domremy, whose cottages, embowered
in greenery, cluster about the little church of Saint Margaret. Many
herds of cattle and sheep are feeding in the pastures between fields
luxuriant with growing crops. Looking back, the eye catches the dusky
summits of the Bois de Chêne,\footnote{Bois de Chêne, or Wood of Oaks,
  is the name of the forest upon the edge of which is Domremy, Joan of
  Arc's native village.} and at the crossroad leading thither stands the
chapel of Saint Catherine.

Between the chapel and the Fairy Tree, and somewhat nearer the latter,
sparkles a bubbling spring whose curative powers were believed in by
those of pious faith in the olden times.

Thus the scene appears under pleasant skies. But when the temperature
suddenly changes, and the cold air rushes down into the valley, its
mists are driven and scattered among the mountainous defiles. At such
times superstitious villagers believe they see the fairies dancing round
the tree, and even the saints of heaven in the wavering shapes of the
mist.

Among the mysterious spots which have invested the neighborhood of
Domremy with such fame and sacredness Bois de Chêne is not the least
famous. One cannot enter its dark recesses without that peculiar feeling
of awe which inspires a solitary wanderer in the presence of nature's
grandeurs,---a feeling which inevitably fills the mind of a
superstitious person with a bewildering array of supernatural fancies.
It was from this very forest that Merlin the wizard predicted the
deliverer of France would come.

Think of a child of susceptible and fanciful nature, fed upon nursery
tales full of superstitions, a child passionately fond of solitary
reveries and fervent appeals to the saints, growing up in such an
environment! Is it remarkable that such a child should see marvels on
the earth and in the air, and the saints themselves in bodily image, and
that she should hear their voices and listen devoutly to angelic music
in the celestial regions?

Just such a child as this sat under the Fairy Tree on a beautiful spring
morning in the year 1424.\footnote{Joan of Arc (Jeanne d'Arc or Darc)
  was born at Domremy, Jan.~6, 1412, and died May 30, 1431. Her father
  was Jacques d'Arc, and her mother Isabelle Romée, illiterate laborers,
  but of good repute. She had three brothers,---Jacques, Pierre, and
  Jean,---and a sister Catharine.} She was a maiden of twelve years, and
was tending a little flock of sheep grazing on the hillside. Even the
casual observer would have noticed her striking appearance, for while
the other girls were frolicking in the meadow below her, she sat leaning
against the tree, gazing fixedly into space, and evidently thinking of
other things than dance, and sport, and herds. Looking more closely into
her lovely oval face and observing its transparent tints and delicate
features, the question would at once suggest itself---How did such a
slight, ethereal creature happen among the children of peasants? Those
wonderful eyes did not merely reveal the self-unconsciousness of the
visionary and the rapture of supernatural contemplation. They were clear
mirrors of the heart, reflecting its inmost recesses and depths. That
heart was the heart of an angel, the heart of a child so innocent it was
impossible not to love her and sympathize with her.

As she sat there, a flock of little birds flew to the tree, filling the
air with the music of their songs. Apparently she did not notice them,
for she neither moved nor changed the expression of her face. They
fluttered down from the tree and hopped about the dreamer, approaching
her more and more nearly, until at last some of them lit on her head and
shoulder. Now for the first time she was conscious of her little guests.

``Ah!'' she exclaimed in a soft melodious voice. ``You are here and I
did not know it.'' She quickly opened a little basket standing near her,
sprinkled some crumbs upon the ground, and watched with childish delight
the liveliness of her tiny companions. Her pleasure, however, was soon
marred by a saucy and envious fellow in the little crowd, who pecked his
neighbor. Chirping sorrowfully, the victim flew to the maiden's feet.

``Alas! alas! poor little bird!'' she exclaimed, the tears coming into
her eyes. She took the little fellow in her lap and caressed him.
``Wait, now, thou envious `wolf,'\,'' she said, addressing the offender.
``Did I not scatter crumbs enough for you all? And did you not know I
would have doubled the amount if that had not been sufficient? You
deserve to be punished for your greediness. Now you shall see how finely
this poor little fellow will fare at his own table.'' Thereupon she
filled her lap from the basket, and the little one ate with a relish,
while the ``wolf'' was not allowed to come near the table, much as he
wished to. Suddenly the flock rose and flew into the branches of the
tree in manifest alarm. Her sheep, which had been feeding below her,
rushed up the hill as fast as they could, and closely huddled together.

``What is the matter?'' cried the maiden, as she cast a hasty glance at
the flying herd. ``What has driven you away from the meadow in such
fright? Holy Catherine! the cruel wolf must be lurking on the edge of
the wood.''

She quickly sprang up, seized her crook, and flew to the Bois de Chêne,
where a wolf was really lying in wait. One who had seen her then would
hardly have recognized the gentle maiden, the dreamer of a moment
before, in this resolute heroine, her eyes flashing with courage.
Wonderful to relate, the beast fled from her. For an instant it
crouched, ready to spring upon her, and then slunk away into the forest.
Thereupon the little heroine went to the neighboring chapel, knelt
before the image of Saint Catherine, and poured out the thankfulness of
her heart in long and fervent prayers. It was her childish belief that
her patron saint had performed a miracle. She did not know that the
beasts of the wood can be intimidated by the firmness and courage of a
fearless person's glance, and that even the lion himself will not attack
such a person unless he is in a frenzy of rage.

As the little one left the chapel the spiritual illumination which
irradiated her face when she sat dreaming under the Fairy Tree again
shone in her beautiful eyes. Her route led her to the miraculous
spring.\footnote{This spring, in the depositions of the witnesses at
  Joan's trial, is always called the ``Well of the Thorn.''} The fresh
green of the bushes and turf allured her. She threw herself down, and
soon was lulled by the gentle plashing of the water into sweet fancies.
For a long time she failed to observe that she had companions who had
come there to drink,---a doe and fawns, who fearlessly approached and
drank the clear water undisturbed. After they had quenched their thirst,
the fawns stood watching the dreamer with their intelligent little eyes
as if they were awaiting friendly recognition from an old acquaintance.
Not receiving it, they sported frolicsomely around her. Suddenly the
charming scene was interrupted. The animals tossed up their heads,
listened intently, and then, as if at a word of command, galloped away
to the forest. A bevy of simple, joyous, sun-browned shepherdesses came
running toward her from the meadow.

``Joan, Joan,'' cried one, ``where are you?''

The maiden rose.

``Aha!'' said the one just speaking, ``she has been listening again to
the murmurs of the spring. Just see how wondrously her eyes glisten!''

At this all of them came up and gazed with a kind of awe at the strange
maiden.

``Well, what do you wish?'' said Joan, gently.

``We have made a wager,'' replied the former speaker. ``See this
beautiful wreath, Joan. After we had woven it we decided it should go to
the winner in a race to the Fairy Tree. Agnes boasted it would be hers.
Margot was just as sure that she would win it. `Ah!' said I; `if Joan
were only here you would not talk this way!' `And why not?' said Agnes.
`Because,' said I, `Saint Catherine always helps her.' `Oh,' interposed
Margot, `I will find Joan and she also shall race.' Then I said, `We
will all search for Joan.' `Yes,' all shouted, `let us find Joan!' And
here we are. Here is the wreath, and there is the Fairy Tree. Will you
run?''

Joan made no reply. She stood absorbed in devotion, and prayed: ``Holy
Catherine, give me the victory, not for my sake, but for thy honor.''

``Joan, do you not hear us?''

``Yes, I am ready.''

Gleefully the maidens formed a line. ``One, two, three,'' a clear voice
counted, and all ran up the hillside. In a few seconds the line was
zig-zag, with Agnes, Margot, and Joan in the lead. Most of the others
gave up the race and followed slowly on, watching the three in eager
suspense. Soon, however, they noticed there was one in the lead, for the
other two had perceptibly fallen back.

``Did I not tell you Joan would win?'' said the one who had first
spoken.

``But there is some witchcraft about it,'' said her neighbor. ``Look at
her, look! Holy Margaret! Her feet do not touch the ground.''

``That is so,'' all said, as they crossed themselves. ``She is flying
through the air.''

It really seemed as if Joan were flying. The mist, the fast-gathering
twilight, and the distance created such an ocular illusion that any
superstitious spectator would have sworn she was flying. All hurried to
the tree, under whose branches the victor was not standing, but devoutly
kneeling. The joyous crowd surrounded her, and no feeling of envy
clouded their joy as they placed the wreath upon her fair head. As the
night was now fast coming on, the girls went homewards with the flocks.
They were all from the village of Domremy.

Joan found Jacques, her father, Pierre, her brother, and Duram Laxart,
her uncle, engaged in earnest conversation with a stranger in the square
in front of the church. A few words which she overheard aroused her
curiosity, and she approached the group and listened.

``I bid you repent,'' said the stranger, ``lest the wrath of Heaven be
visited upon you, for all the misfortunes of this land are divine
punishments for the sins of the Court and the King's kindred.''

``Oh, oh, holy father,'' said one, ``that would be very sad.''

``What do you mean by that, my son?''

``I mean it would be very sad for Heaven to punish poor people who have
done no wrong, for the wrongdoings of the Court.''

``Go home, thou son of Belial who doubtest that which the Spirit reveals
to thee through my lips. Shut thyself up in thy chamber, and three times
repeat seven paternosters, that thy soul may be released from the bonds
of doubt, for doubt is the work of the devil, who is already stretching
out his claws to seize thee.''

``But, holy father---''

``Be quiet, Gamoche,'' interposed another villager. ``Do not interrupt
the holy father. He will explain it all to us.''

``Yes, yes,'' cried the others, ``he will explain everything.''

``Well, then, listen to me, children,'' resumed the stranger. ``But,
holy Mother of God, where shall I begin? The list of the sins of this
Court is so long that if I should go back a century, even then it would
not be the beginning. I will confine myself to the recent ones, which
must be more or less familiar to you all. Have you heard about the last
King, Charles the Sixth?''\footnote{Charles the Sixth was born at Paris
  in 1368, and died in 1422. He reigned forty-two years, but became
  deranged in 1392, and the Duke of Orleans, his brother, gained the
  ascendancy. It was his Queen, Isabella, who prepared the way for the
  treaty of Troyes, which was to make Henry the Fifth of England King of
  France on Charles's death.}

``Why should we not have heard? He died insane only two years ago.''

``Yes, insane. He had a few lucid moments after 1392, in which he
recognized in some measure the profligacy of the administration. The
whole royal family, with but few exceptions, acted as if they were
insane. First of all, there was the Queen, the notorious Isabella of
Bavaria, who was as much a stranger to the nobility of human nature as
she was to the divine. Her every purpose and act had no higher motive
than the gratification of her own desires and the discovery how to
accomplish them. It would have mattered nothing to her if a sea of blood
had been shed, if only her interests were advanced. There was the Duke
Louis of Orleans,\footnote{After the derangement of his brother, Louis
  assumed the regency in opposition to the Duke of Burgundy. He was
  assassinated by the latter in 1407.} brother of the insane King, who
pandered to Isabella's profligacy and lust of power, finally seized the
reins of sovereignty, and plunged the state into direst confusion. There
were the King's uncles, the dukes of Bourbon, Berry, Burgundy, and
Anjou, all alike avaricious and ambitious for power, who lashed the Duke
of Orleans and the King with the scourge of war, murdered their
subjects, and ravaged the country. Then came numerous factions which
contended with one another, one for this, and one for that, and finally
almost countless great and little lords, robber barons, who, pretending
to espouse the cause of one party, harried the districts of others,
leaving a trail of pillage and blood. To complete the burden of
wretchedness, King Henry the Fifth sent his Englishmen, those hereditary
enemies of France, across the Channel. In alliance with the turbulent
dukes, particularly those of Burgundy and Brittany, they advanced
victorious, captured one place after another, and at last even Rouen and
Paris, so that few provinces were left to the unfortunate King.
Frightful confusion followed when this King died in 1422. Henry the
Fifth, to be sure, died in the same year, but his field marshal, the
Duke of Bedford, guardian of young Henry the Sixth,\footnote{Henry the
  Sixth was crowned King of France in 1430, but lost all his French
  possessions except Calais, owing to the successes of Joan of Arc. The
  Duke of Bedford was his uncle.} did not abandon the field. The
infamous treaty of Troyes gave him the semblance of right.''

``How so, holy father?'' interposed one of the villagers.

``Be quiet,'' replied another. ``You ought to have known that Queen
Isabella, out of hate and revenge against her youngest son Charles, who,
after the death of his brother the Dauphin, was crown prince, concluded
that treaty with England whereby the French royal family was barred from
the succession and the King of England was declared successor of Charles
the Sixth.''

``Oh, the disgrace! Oh, the shame!'' several exclaimed.

``And this poor Dauphin,'' continued the former speak-er, ``spent a
joyless youth, in which his unnatural mother often forced him as well as
his father to suffer the pangs of hunger; and yet, poor, weak, and
throneless as he is, he is still ready to struggle for that throne which
is his birthright as Charles the Seventh. Is this not so, holy father?''

``Certainly, certainly, God's pity,'' replied the stranger. ``He should
rule by his own and by divine right. The treaty of Troyes cannot prevent
it. But where is the hero who will lead him to coronation at Rheims?
Alas, only miraculous interposition can save him from ruin.''

``Saint Catherine,'' sighed a gentle voice.

``Joan!'' exclaimed Jacques, as he recognized his daughter, ``what are
you doing here? Go home.''

``Not yet, father Jacques,'' said the stranger. ``Let her stay. Do you
not know that the prayers from a pure child's heart are heard by the
dear saints? And,'' he added, ``I have never seen eyes so full of
innocence and piety as hers.''

``Ah!'' replied Jacques, ``of what use are the prayers of a child when
the whole country lies helpless?''

``Are you also an unbeliever?'' replied the stranger. ``Know you not
that the great God can manifest Himself in a little child?''

These were the last words of the conversation which Joan heard. She
suddenly disappeared, but she did not go home. She wended her way to the
church, which was always open. Never had her heart been so troubled and
full of strange longings, never had she been so powerfully moved to hold
communion with her saint. It was not so much the desire to make a votive
offering of her wreath as it was the unspeakable sorrow of the
fatherland and the wretched plight of the poor Dauphin that urged her to
this sacred spot. And was this strange? If her sympathetic nature made
her shed tears over the slight suffering of a bird, how much more would
it force her to weep over the story of universal misfortune which she
had just heard! Why should not the courage with which she had defended
her sheep from the wolf display itself now even more decidedly? And why
should she not believe in her very soul that her favorite saint would
perform a miracle of rescue?

``Oh, were I only a man!'' she sighed from the depth of her heart. ``Oh
that I could clothe my limbs in armor and wield the sword for the right!
I would ask for nothing better in life. No sacrifice would be too great
to accomplish it. Then, surely, the beloved saints would not refuse to
help me.''

In such a spirit she entered the sacred house. It was empty. The shadows
of evening, mingling with the clouds of incense smoke which still
lingered in the church, were intensified by the feeble light of a small
lamp. She thrilled with sacred awe as she advanced through the
mysterious gloom. In her exalted mood it seemed to her that Saint
Catherine smiled, as with trembling hand she placed the wreath upon her
altar. In transports of sorrow and gratitude, of divine trust, and of
overwhelming desire for action, she knelt at the altar, and her soul
ascended to the celestial abodes. She knew no prayers except the Lord's
Prayer, the Credo, and the Ave Maria, but the more she repeated them the
more completely was she spiritually absorbed.

Thus little by little she sank into that species of ecstasy in which the
ordinary spiritual functions are suspended and there remain only the
sacred feeling of heavenly contemplation and the free play of the fancy.
It is a condition which differs from actual dreaming only in its danger,
for there is danger that this ecstatic feeling once aroused may become
real, and its possessor may behold illusive pictures of the fancy. The
enthusiast may believe he sees real objects and hears actual voices. He
may believe them to be messages from heaven, never asking himself
whether such fancies will stand the test of reason. Because of ecstasies
like these, deeds have been committed which have darkened the page of
history with everlasting shame. But when these ecstasies arise from
exalted moral ideas they may achieve results which are far beyond mere
human strength and secure imperishable fame for the enthusiast.

Thus it was with this simple child praying at the altar. In her ecstatic
fancy she saw the roof of the church open, and her favorite Saints
Catherine and Margaret floating down through the clouds of incense. She
heard them saying, ``Keep thy heart unsullied, Joan, for Heaven has
chosen thee as the champion of France.''

The vision disappeared. The dream was over. But in that instant the
career of this child was determined. She was the subsequent Maid of
Orleans.

\threeast
