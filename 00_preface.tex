\addchap{\scshape preface}
\setcounter{fopagecnt}{2}

\drop The life story of Joan of Arc, as told in this volume, closely
follows the historical facts as well as the official records bearing
upon her trial and burning for ``heresy, relapse, apostasy, and
idolatry.'' It naturally divides into two parts. First, the simple
pastoral life of the shepherd maiden of Domremy, which is charmingly
portrayed; the visions of her favorite saints; the heavenly voices which
commissioned her to raise the English siege of Orleans and crown the
Dauphin; her touching farewell to her home; and, secondly, the part she
played as the Maid of Orleans in the stirring events of the field; the
victories which she achieved over the English and their Burgundian
allies; the raising of the siege; the coronation of the ungrateful
Dauphin at Rheims; her fatal mistake in remaining in his service after
her mission was accomplished; her capture at Compiègne; her infamous
sale to the English by Burgundy; her more infamous trial by the corrupt
and execrable Cauchon; and her cruel martyrdom at the stake. Another
story, the abduction of Marie of Chafleur, her rescue by Jean Renault,
and their final happiness, is closely interwoven with the movement of
the main story, and serves to lighten up the closing chapters. This
episode is pure romance of an exciting nature; but the life of the Maid
of Orleans is a remarkably faithful historical picture, which is all the
more vivid because the characters are real. In this respect it resembles
nearly all the volumes in the numerous German ``libraries for youth.''
They are stories of real lives, concisely, charmingly, and honestly
told, and adhere so closely to fact that the reader forms something like
an intimate personal acquaintance with the characters they introduce.

\begin{flushright}
\textsc{G. P. U.}
\end{flushright}

\begin{flushright}
\textit{Chicago, 1904.}
\end{flushright}
